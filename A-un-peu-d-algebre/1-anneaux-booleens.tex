\section{Anneaux Booléens}

$F_2$ désignera le corps à deux éléments $\mathbb{Z}/2\mathbb{Z}$. Un \textit{anneau Booléen} est un anneau commutatif unitaire $(A, +, *)$ dans lequel, pour tout $x\in A, x*x=x$ ($F_2$ est un exemple d'anneau booléen).
\medskip

Si $E$ est un ensemble, on note $\mathcal{P}(E)$ l'ensemble des sous-ensembles finis de $E$ et $\mathcal{AB}[E]$ l'ensemble des fonctions de $\mathcal{P}(E)$ dans $F_2$ qui sont nulles sauf pour un nombre fini d'éléments de $\mathcal{P}(E)$. $\mathcal{AB}[E]$ est muni de l'addition et de la multiplication : $(f+g)(x)=f(x)+g(x)$ et $(fg)(x)=f(x)g(x)$. $\mathcal{AB}[E]$ muni de ces opérations est aussi un anneau booléen : c'est l'anneau booléen \textit{engendré} par $E$.
\medskip

Dans tout le problème, $E$ sera supposé fini et de cardinal $n$.