\renewcommand{\SourceFile}{2-jouer-avec-les-mots/src/2-7.ml}

\section{Code correcteur de Hamming}

Nous voulons transmettre des messages composés de 0 et de 1 (chiffres binaires). Comme une erreur peut se produire lors de la transmission, nous utilisons un code correcteur de Hamming$(n,m)$ où $n=2^r-1$ et $m=n-r$ qui encode tout mot $V$ de $m$ chiffres binaires en un mot de code $C$ composé de $n$ chiffres binaires. Soit $V=v_1v_2...v_m$ un mot du message et $C=c_1c_2...c_r$ le mot de code correspondant. $C$ est construit de la façon suivante :
\medskip

Si $2^{k-1} < i < 2^{k}$, ($k > 0$) alors $c_i=v_{i-k}$\\
Si $i=2^k$ avec $k=0...r$ alors $c_i=0$ si la somme $\displaystyle\sum_{\substack{j=1\\ j \neq 2^k \textrm{ et } j_k=1}}^{n}c_j$ est paire, $c_i=1$ sinon.

Avec $\displaystyle j=\sum_{k=0}^{r-1}j_k2^k$ (écriture binaire de $j$).\\
Pour simplifier, nous prendrons des tableaux d'entiers pour stocker $V$ et $C$.

\Q
Le passage du message $V$ au message encodé $C$ revient à un produit matrice-vecteur tel que : $C=AV$, où $A$ est une matrice $(n \times m)$. Écrire la matrice $A$ pour $n=31$ et $m=26$.\\
Écrire une fonction OCaml \texttt{gen\_mat} qui, étant donné deux nombres $n$ et $m$, génère la matrice génératrice $A$ du code de Hamming$(n,m)$.

\Q
Écrire une fonction OCaml qui, étant donné un mot $V$ de $m$ chiffres binaires et une matrice $A$, renvoie le mot de code $C$ correspondant.

\Q
À la réception du message encodé $C'$ ($C$ avec peut-être une erreur de transmission), nous désirons reconstituer le message $V$. Pour ceci, nous vérifions que le message $C'$ est un mot du code défini à la question précédente. Construire, pour $m=26$ et $n=31$, la matrice $(5 \times 31)$ de contrôle de parité $H$ telle que :
\medskip

$H \times C' = S$ où $S$ est un vecteur de dimension 5 tel que :
\begin{itemize}
    \item Si $C'$ est un mot du code alors $S$ est le vecteur nul
    \item Sinon, $S$ donne l'écriture binaire de l'indice de l'élément erroné de $C'$
\end{itemize}
\medskip

Écrire une fonction OCaml \texttt{cont\_mat} qui, étant donné deux nombres $n$ et $r$ génère la matrice de contrôle de parité $H$ du code de Hamming$(n,n-r)$.

\Q
Écrire une fonction OCaml \texttt{syndrome} qui, étant donné le mot de code $C'$ et une matrice de contrôle de parité $H$ du code de Hamming$(n,n-r)$, renvoie l'indice de l'élément erroné de $C'$ ou $-1$ si $C'$ est sans erreur.
\medskip

Écrire une fonction OCaml \texttt{decode} qui, étant donné le mot de code $C'$ et une matrice de contrôle de parité $H$ du code de Hamming$(n,n-r)$, renvoie le message original $V$ à partir de $C'$ après avoir au besoin corrigé l'erreur détectée.

\Corrige

\Q
Pour $n=31$ et $m=26$, nous obtenons la matrice $A$ suivante :

\lstinputlisting[linerange={1-31}]{2-jouer-avec-les-mots/src/2-7.txt}


Nous proposons la fonction \texttt{gen\_mat} suivante qui utilise le fait que la matrice contient beaucoup de 0 :

\lstinputlisting[linerange={1-29}]{\SourceFile}

\Q
C'est un simple produit matrice-vecteur.

\lstinputlisting[linerange={42-52}]{\SourceFile}

\Q
La matrice $(5 \times 31)$ de contrôle de parité $H$ pour $m=26$ et $n=31$ est :

\lstinputlisting[linerange={33-37}]{2-jouer-avec-les-mots/src/2-7.txt}
\medskip

La fonction \texttt{cont\_mat} qui, étant donné deux nombres $n$ et $r$ génère la matrice de contrôle de parité $H$ du code de Hamming$(n,n-r)$ peut s'écrire :

\lstinputlisting[linerange={54-65}]{\SourceFile}

\Q
Nous présentons ici une fonction \textit{syndrome} qui, étant donné le mot de code $C'$ et une matrice de contrôle de parité $H$ du code de Hamming$(n,n-r)$, renvoie l'indice de l'élément erroné ou $-1$ si $C'$ est sans erreur.
\medskip

Nous avons là encore un simple produit matrice-vecteur :

\lstinputlisting[linerange={67-83}]{\SourceFile}

La fonction \texttt{decode} définie dans l'énoncé peut s'écrire comme suit.

\lstinputlisting[linerange={85-101}]{\SourceFile}

\Fin
