\renewcommand{\SourceFile}{2-jouer-avec-les-mots/src/2-2.ml}

\section{Pliage de papier}

On prend une feuille de papier et on la plie $n$ fois dans le sens vertical, en repliant à chaque fois la moitié droite sur la gauche. Les plis de la feuille, une fois redépliée, sont une suite de creux et de bosses. Voir figure :
\medskip

\begin{tikzpicture}[scale=1.1]
    \draw (0,0) rectangle (4,2);

    \node[font=\ttfamily, below=.1cm of {1.2,0}] {étape 0};
\end{tikzpicture}\hspace{2cm}
\begin{tikzpicture}[scale=1.1]
    \draw (0,0) rectangle (4,2);
    \draw (2,0) -- (2,2);

    \node[font=\ttfamily, below=.1cm of {1.2,0}] {étape 1};
\end{tikzpicture}
\medskip

\begin{tikzpicture}[scale=1.1]
    \draw (0,0) rectangle (4,2);
    \draw (1,0) -- (1,2);
    \draw (2,0) -- (2,2);
    \draw[ultra thick] (3,0) -- (3,2);

    \node[font=\ttfamily, below=.1cm of {1.2,0}] {étape 2};
\end{tikzpicture}\hspace{2cm}
\begin{tikzpicture}[scale=1.1]
    \draw (0,0) rectangle (4,2);
    \draw (.5,0) -- (.5,2);
    \draw (1,0) -- (1,2);
    \draw [ultra thick](1.5,0) -- (1.5,2);
    \draw (2,0) -- (2,2);
    \draw (2.5,0) -- (2.5,2);
    \draw[ultra thick] (3,0) -- (3,2);
    \draw[ultra thick] (3.5,0) -- (3.5,2);

    \node[font=\ttfamily, below=.1cm of {1.2,0}] {étape 3};
\end{tikzpicture}
\vspace{.4cm}

\hspace{.6cm}
\begin{tikzpicture}
    \draw (0,0) -- (0,.6);
    \node[font=\ttfamily, right=.2cm of {0,.3}] {$\leftarrow$ creux};

    \draw[ultra thick] (4,0) -- (4,.6);
    \node[font=\ttfamily, right=.2cm of {4,.3}] {$\leftarrow$ bosse};
\end{tikzpicture}

\Q
Combien y a-t-il de plis à la n-ième étape ?

\Q
On représente chaque étape du pliage par un mot. Un creu est codé par un 0 et une bosse par un 1. Ainsi : $w_0=\varepsilon$ (mot vide)\\
$w_1=0$\\
$w_2=001$\\
$w_3=0010011$\\
\dots
\medskip

Montrer que $w_i$ est toujours un préfixe de $w_{i+1}$, c'est à dire que le début de $w_{i+1}$ coïncide avec $w_i$.

\Q
Donner un algorithme de construction de $w_n$ à partir de $w_{n-1}$. Écrire une fonction de calcul de $w_n$. On prendra comme entrée $n$ et on renverra une liste \texttt{w} d'entiers remplie adéquatement.

\Q
Écrire une fonction qui prend pour entrée un entier $n$ et renvoie une liste contenant la représentation binaire de $n$, poids fort en tête.

\Q
Les mots de la suite de pliage étant préfixes les uns des autres, on peut considérer le mot infini $w$ dont ils sont tous préfixes. Proposer un algorithme qui prend pour entrée la représentation binaire de $n$ et donne en sortie le $n$-ième bit de $w$.

\Corrige

\Q
Si on considère