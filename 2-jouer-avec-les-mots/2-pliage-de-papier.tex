\renewcommand{\SourceFile}{2-jouer-avec-les-mots/src/2-2.ml}

\section{Pliage de papier}

On prend une feuille de papier et on la plie $n$ fois dans le sens vertical, en repliant à chaque fois la moitié droite sur la gauche. Les plis de la feuille, une fois redépliée, sont une suite de creux et de bosses. Voir figure :
\medskip

\begin{tikzpicture}[scale=1.1]
    \draw (0,0) rectangle (4,2);

    \node[font=\ttfamily, below=.1cm of {1.2,0}] {étape 0};
\end{tikzpicture}\hspace{2cm}
\begin{tikzpicture}[scale=1.1]
    \draw (0,0) rectangle (4,2);
    \draw (2,0) -- (2,2);

    \node[font=\ttfamily, below=.1cm of {1.2,0}] {étape 1};
\end{tikzpicture}
\medskip

\begin{tikzpicture}[scale=1.1]
    \draw (0,0) rectangle (4,2);
    \draw (1,0) -- (1,2);
    \draw (2,0) -- (2,2);
    \draw[ultra thick] (3,0) -- (3,2);

    \node[font=\ttfamily, below=.1cm of {1.2,0}] {étape 2};
\end{tikzpicture}\hspace{2cm}
\begin{tikzpicture}[scale=1.1]
    \draw (0,0) rectangle (4,2);
    \draw (.5,0) -- (.5,2);
    \draw (1,0) -- (1,2);
    \draw [ultra thick](1.5,0) -- (1.5,2);
    \draw (2,0) -- (2,2);
    \draw (2.5,0) -- (2.5,2);
    \draw[ultra thick] (3,0) -- (3,2);
    \draw[ultra thick] (3.5,0) -- (3.5,2);

    \node[font=\ttfamily, below=.1cm of {1.2,0}] {étape 3};
\end{tikzpicture}
\vspace{.4cm}

\hspace{.6cm}
\begin{tikzpicture}
    \draw (0,0) -- (0,.6);
    \node[font=\ttfamily, right=.2cm of {0,.3}] {$\leftarrow$ creux};

    \draw[ultra thick] (4,0) -- (4,.6);
    \node[font=\ttfamily, right=.2cm of {4,.3}] {$\leftarrow$ bosse};
\end{tikzpicture}

\Q
Combien y a-t-il de plis à la n-ième étape ?

\Q
On représente chaque étape du pliage par un mot. Un creux est codé par un 0 et une bosse par un 1. Ainsi : $w_0=\varepsilon$ (mot vide)\\
$w_1=0$\\
$w_2=001$\\
$w_3=0010011$\\
\dots
\medskip

Montrer que $w_i$ est toujours un préfixe de $w_{i+1}$, c'est à dire que le début de $w_{i+1}$ coïncide avec $w_i$.

\Q
Donner un algorithme de construction de $w_n$ à partir de $w_{n-1}$. Écrire une fonction de calcul de $w_n$. On prendra comme entrée $n$ et on renverra une liste \texttt{w} d'entiers remplie adéquatement.

\Q
Écrire une fonction qui prend pour entrée un entier $n$ et renvoie une liste contenant la représentation binaire de $n$, poids fort en tête.

\Q
Les mots de la suite de pliage étant préfixes les uns des autres, on peut considérer le mot infini $w$ dont ils sont tous préfixes. Proposer un algorithme qui prend pour entrée la représentation binaire de $n$ et renvoie le $n$-ième bit de $w$.

\Corrige

\Q
Si on considère les intervalles entre les plis (y compris les côtés de la feuille), on a 1 intervalle, puis 2, puis 4, etc. ; à chaque étape, chaque intervalle est coupé en deux et le nombre d'intervalles double. Après $n$ étapes, il y a donc $2^n$ intervalles, et le nombre de plis est $2^n-1$.

\Q
Déchirons le papier le long du pli de la première étape et débarrassons-nous de la moitié de droite : la $k$-ième étape de pliage du papier initial est comme la $(k-1)$-ième étape de pliage du demi-papier, et donc $w_{k-1}$ forme la moitié gauche de $w_k$.

\Q
Le pli du milieu est un creux et donc donne un 0 au milieu de $w_n$. Le demi-papier droit est plié, après la première étape, comme le demi-papier gauche, sauf qu'il est tourné dans l'autre sens. Soit $\textrm{r}(w)$ le mot obtenu à partir d'un mot $w$ en lisant les chiffres de droite à gauche, et $\textrm{c(w)}$ le mot obtenu à partir de $w$ en remplaçant chaque 0 par un 1 et chaque 1 par un 0 : on a la relation de récurrence
\[
    w_n = w_{n-1}\,0\,\textrm{c}(\textrm{r}(w_{n-1}))
\]
De cette relation, se déduit facilement la fonction permettant de construire $w_n$.

\lstinputlisting[linerange={1-5}]{\SourceFile}

\Q
Question classique et accessible à tous.

\lstinputlisting[linerange={7-12}]{\SourceFile}

\Q
On peut trouver un autre point de vue pour construire $w_k$ par récurrence à partir de $w_{k-1}$ : on intercale un nouveau pli entre chaque paire de plis consécutifs de $w_{k-1}$, et ce pli est alternativement 0 ou 1, le premier étant 0. Ainsi : $w_k=0x1x0x...x1$, si $w_{k-1}=xxx...xx$. Donc le $n$-ième bit de $w$ est facile à trouver si $n$ est impair : c'est 0 si $n$ est congru à 1 modulo 4 et c'est 1 si $n$ est congru à 3 modulo 4. Si $n$ est pair, on divise $n$ par deux et on remarque que le $n$-ième bit de $w_k$ est le $(n/2)$-ième bit de $w_{k-1}$, donc si $n/2$ est impair, il suffit encore une fois de tester si $n/2$ est congru à 1 ou à 3 modulo 4. Si $n/2$ est est pair, on redivise par deux, et ainsi de suite. Lorsque $n$ est écrit en binaire, on regarde son bit le plus à droite (celui de poids le plus faible), puis le bit immédiatement à sa gauche, et ainsi de suite jusqu'à trouver un bit égal à 1. Soit $x$ le $n$-ième bit de $w$. On a :\\
Si $n=1~0~0~...~0$, alors $x=0$.\\
Si $n=*~*~...~*~0~1~0~0~...~0$, alors $x=0$.\\
Si $n=*~*~...~*~1~1~0~0~...~0$, alors $x=1$.\\
Car un nombre se terminant par 0~1 est congru à 1 modulo 4 et un nombre se terminant par 1~1 à 3 modulo 4.
Ceci donne un programme extrêmement simple.

\lstinputlisting[firstline=14]{\SourceFile}
\medskip

Des études du genre de celle faite dans cet exercice peuvent servir à montrer des propriétés d'algébricité ou de transcendance du nombre étudié. C'est actuellement un domaine actif de recherche en France. On obtient une suite de courbes $(C_n)$ approximant une courbe de Peano (c'est-à-dire une courbe qui remplit le plan), définie à partir de $(w_n)$, en dessinant un segment de longueur $1/2^{n/2}$ pour chaque bit lu, et en tournant à droite (de $+\pi/2$) ou à gauche (de $-\pi/2$) à chaque pas, selon que le bit lu est 0 ou 1. Le premier segment est dessiné dans la direction $\pi/4$. La figure suivante illustre les premiers pas :

\begin{center}
\begin{tikzpicture}[scale=.8]
    \draw (0,0) -- (3,0);

    \node[font=\ttfamily, above=.1cm of {1.2,0}] {C0};
\end{tikzpicture}\hspace{2cm}
\begin{tikzpicture}[rotate=45, scale=.8]
    \draw (0,2.1) -- (2.1,2.1);
    \draw (2.1,2.1) -- (2.1,0);

    \node[font=\ttfamily, below=.1cm of {1.6,1.4}] {C1};
\end{tikzpicture}
\vspace{1cm}

\begin{tikzpicture}[scale=.8]
    \draw (0,0) -- (0,1.5);
    \draw (0,1.5) -- (1.5,1.5);
    \draw (1.5,1.5) -- (1.5,0);
    \draw (1.5,0) -- (3,0);

    \node[font=\ttfamily, above=.1cm of {1,0}] {C2};
\end{tikzpicture}\hspace{2cm}
\begin{tikzpicture}[rotate=45, scale=.8]
    \draw (0,2.2) -- (0,3.3);
    \draw (0,3.3) -- (1.1,3.3);
    \draw (1.1,3.3) -- (1.1,2.2);
    \draw (1.1,2.2) -- (2.2,2.2);
    \draw (2.2,2.2) -- (2.2,1.1);
    \draw (2.2,1.1) -- (1.1,1.1);
    \draw (1.1,1.1) -- (1.1,0);
    \draw (1.1,0) -- (2.2,0);

    \node[font=\ttfamily, below=.1cm of {.9,2}] {C3};
\end{tikzpicture}
\end{center}
\medskip

\Fin
