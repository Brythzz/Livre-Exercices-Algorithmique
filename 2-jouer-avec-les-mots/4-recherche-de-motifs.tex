\renewcommand{\SourceFile}{2-jouer-avec-les-mots/src/2-4.ml}

\section{Recherche de motifs}

Un mot $u$ est une suite $(u_1, u_2, ..., u_n)$ de lettres, l'entier $n$ est la longueur du mot $u$. La suite vide correspond au mot vide, noté $\varepsilon$, de longueur 0. La concaténation de deux mots $u=(u_1, u_2, ..., u_n)$ et $v=(v_1, v_2, ..., v_k)$, notée $uv$, est simplement le mot $(u_1, u_2, ..., u_n, v_1, v_2, ..., v_k)$. Il est facile de vérifier que la concaténation est une opération associative pour laquelle le mot vide est élément neutre à gauche et à droite. Le mot $v$ est dit facteur ou motif d'un mot $u$ s'il existe des mots $g$ et $d$ tels que $u=gvd$. Si le mot $g$ (resp. $d$) est la suite vide, on dit que $v$ est un préfixe (resp. suffixe) de $u$.

\Q
Modéliser la situation décrite. Écrire un algorithme qui, étant donnés deux mots $u$ et $v$, vérifie si $v$ est un facteur de $u$. Quel est le nombre maximum de comparaisons effectuées par votre algorithme ?

\Q
Si $v$ n'est pas le mot vide, on note $Bord(v)$ le plus grand mot distinct de $v$ qui soit à la fois préfixe et suffixe de $v$. Par exemple, si $v=abacaba$, $Bord(v)=aba$. Soit $v=(v_1, v_2, ..., v_k)$. Pour tout $i$ entre 1 et $k$, on définit $B(i)$ comme étant la longueur de $Bord(v_1 ... v_k)$.
\smallskip

Utiliser cette fonction $B$ pour écrire un nouvel algorithme vérifiant si un mot $v$ est facteur d'un mot $u$. Quel est le nombre maximum de comparaisons effectuées par ce nouvel algorithme ?

\Q
Soit $v=(v_1, v_2, ..., v_k)$ et $a$ une lettre. On admet que $Bord(va)$ est le plus long préfixe de $v$ qui soit dans $\{\varepsilon, Bord(v)a,Bord^2(v)a, ..., Bord^k(v)a\}$ (en posant $Bord(\varepsilon)=\varepsilon$). Déduire de ce résultat un algorithme qui, étant donné un mot $v=(v_1, v_2, ..., v_k)$, calcule successivement $B(1), ..., B(k)$.

\newpage

\Corrige

\Q
La modélisation suggérée consiste à représenter les mots par des chaînes de caractères.
\medskip

L'algorithme naïf, mais naturel, consiste à comparer $v$ avec les facteurs de $u$ de même longueur que $v$ en commençant en posotion 0, puis 1, 2... jusqu'au succès ou à la fin de $u$.

\lstinputlisting[linerange={1-12}]{\SourceFile}

Chaque appel à la fonction \texttt{String.equal} amène à au plus $k$ comparaisons. Le nombre d'appels à \texttt{String.equal} étant d'au plus $(n-k+1)$, le nombre total de comparaisons effectuées par cet algorithme est au plus $k(n-k+1)$. Le maximum est atteint pour les mots de la forme $u=a^{n-1}b$ et $v=a^{k-1}b$.

\Q
La primitive Caml \texttt{equal} compare les chaînes de caractère passées en argument octet par octet (caractère par caractère). On donne la code OCaml équivalent suivant :

\lstinputlisting[linerange={14-24}]{\SourceFile}

Supposons que l'appel à \texttt{String.equal} renvoie false lors de la comparaison entre $u_{i+j-1}$ et $v_j$. On a alors la situation suivante :
\begin{center}
    \begin{tabular}{ c c c c c }
        $u_i$ & $u_{i+1}$ & ... & $u_{i+j-2}$ & $u_{i+j-1}$ \\
        = & = & & = & $\neq$ \\
        $v_1$ & $v_2$ & ... & $v_{j-1}$ & $v_{j}$
    \end{tabular}
\end{center}

On va donc ensuite comparer la chaîne de caractères $(u_{i+1}, u_{i+2}, ..., u_{i+j})$ à $v$. Pour que cet appel renvoie $true$, il faut en particulier que $v_1=v_2$, ..., $v_{j-2}=v_{j-1}$. Ces égalités signifient exactement que $Bord(v_1...v_{j-1})=v_1...v_{j-2}$. Ainsi, si $Bord(v_1...v_{j-1}) \neq v_1...v_{j-2}$, il est inutile d'appeler \texttt{String.equal} avec la chaîne de caractères $(u_{i+1}$, $u_{i+2}$, ..., $u_{i+j})$. De façon identique, on voit que si $Bord(v_1...v_{j-1}) \neq v_1...v_{j-3}$, l'appel à \texttt{String.equal} avec la chaîne de caractères $(u_{i+2}, u_{i+3}, ..., u_{i+j+1})$ est inutile. En répétant le raisonnement, il apparaît que le premier appel qui ne soit pas voué à l'échec est celui où $u_{i+j-1}$ est comparé à $v_{1+B(j-1)}$. Cette remarque conduit alors au nouvel algorithme suivant, où on suppose la fonction \texttt{b} correspondant à $B$ définie :

\lstinputlisting[linerange={51-64}]{\SourceFile}

Le nombre maximal de comparaisons peut être évalué comme suit. Appelons \og positif \fg{} un test $u_i=v_j$ évalué à $true$ et négatif un test $u_i=v_j$ évalué à $false$. Lors d'un test positif, $i$ augmente, il y a donc au plus $n$ tests positifs. Lors d'un test négatif, la quantité $i-j$ augmente : elle vaut 0 au commencement de l'algorithme et au plus $n$ lorsque l'algorithme se termine. Ainsi, il y a également au plus $n$ tests négatifs. En conclusion, ce second algorithme effectue donc au plus $2n$ comparaisons.

\Q
On a bien sûr $B(1)=0$ et $B(2)=1$ si $v_1=v_2$, 0 sinon.
\smallskip

Supposons calculés $B(1)$, ..., $B(h-1)$. D'après l'indication donnée dans l'énoncé, $B(h)$ est la longueur du plus long préfixe de $v_1...v_h$ qui soit dans
\[
    \{\varepsilon, Bord(v_1...v_{h-1})v_h,Bord^2(v_1...v_{h-1})v_h, ...\}
\]
Il est clair que la longueur des mots $Bord^i(v_1...v_{h-1})v_h$ décroit (au sens large) lorsque $i$ croit. Ainsi, il faut d'abord tester si $Bord(v_1...v_{h-1})v_h$ est un préfixe de $v_1...v_h$, puis si ce n'est pas le cas, si $Bord^2(v_1...v_{h-1})v_h$ est un préfixe de $v_1...v_h$, etc. En remarquant que $Bord(v_1...v_{h-1})v_h$ est un préfixe de $v_1...v_h$ si et seulement si $v_h=v_{1+B(h-1)}$, on est amenés naturellement à l'algorithme suivant, renvoyant le tableau $[B(1), B(2), ..., B(k)]$ :

\lstinputlisting[linerange=26-46]{\SourceFile}

En considérant la quantité $h-j$, on obtient par un argument similaire à celui de la question 2 que le nombre de comparaisons effectuées est au plus $2k$.
\bigskip

L'indication de l'énoncé : $Bord(va)$ est le plus long préfixe de $v$ qui soit dans \newline $\{\varepsilon, Bord(v)a,Bord^2(v)a, ..., Bord^k(v)a\}$ se démontre comme suit.\\
On dit que $w$ est un bord de $v$ si $w$ est à la fois un préfixe et un suffixe de $v$. Soit $z$ un bord de $va$. Si $z \neq \varepsilon$, $z = z'a$, où $z'$ est un bord de $v$. Ou bien $z'=Bord(v)$, ou bien $z'$ est un bord de $Bord(v)$. Par induction, il vient immédiatement que $z'$ est de la forme $Bord^i(v)$ pour un certain $i$ tel que $1 \leq i \leq k$. Ainsi, $z$ est dans l'ensemble $\{\varepsilon, Bord(v)a,Bord^2(v)a, ..., Bord^k(v)a\}$.
\smallskip

Réciproquement, tout mot de cet ensemble est clairement suffixe de $va$, donc un bord de $va$ s'il est préfixe de $va$.
\bigskip

\Fin
