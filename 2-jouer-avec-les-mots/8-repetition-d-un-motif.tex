\renewcommand{\SourceFile}{2-jouer-avec-les-mots/src/2-8.ml}

\section{Répétition d'un motif}

Nous disposons d'un alphabet $\Sigma$. Nous nous intéressons aux mots ne possédant pas deux facteurs consécutifs égaux, autrement dit les mots sans fecteurs carrés : ils ne peuvent pas s'écrire $rs^2t$ avec $|s|\geq1$.

\Q
Montrer que si $\Sigma$ est réduit à deux éléments alors tout mot de quatre lettres ou plus possède un facteur carré.

\Q
Écrire une fonction booléenne qui renvoie \texttt{true} si le mot entré ne comporte pas de facteur carré et \texttt{false} sinon.
\smallskip

Préciser le nombre maximum d'itérations.

\Q
On considère l'alphabet $\Sigma=\{0,1\}$ et le morphisme $\sigma$ défini par $\sigma(0) = 01$ et $\sigma(1) = 10$.
\vspace{-12pt}
\smallskip

Nous avons $\sigma(mm')=\sigma(m)\sigma(m')$ où $m$ et $m'$ sont deux mots et $mm'$ est la concaténation de $m$ et $m'$.
\smallskip

Nous pouvons ainsi générer la suite : $0 \rightarrow 01 \rightarrow 0110 \rightarrow 01101001 \rightarrow ...$ en appliquant $\sigma$ successivement. Le mot infini obtenu est appelé la suite de Thue-Morse.
\medskip

Écrire une fonction permettant de générer une telle suite.
\medskip

Montrer qu'il est impossible d'avoir une sous-suite de la forme $\omega\omega x$ où $\omega$ est un mot et $x$ la première lettre de $\omega$.

\Q
Construire la suite des nombres de 1 compris entre deux 0 de la suite précédente.
\medskip

Montrer que l'alphabet $\{0,1,2\}$ est suffisant pour l'écrire, et qu'elle ne possède pas de facteurs carrés.
\medskip

Déterminer le morphisme correspondant.

\Corrige

\Q
Notons $\Sigma=\{a,b\}$. Tout mot de plus de trois lettres sur l'alphabet $\Sigma$ et sans facteur carré possède l'un des deux préfixes suivants : $aba$ ou $bab$ (les six autres facteurs possibles comportent tous un carré). S'il possède au moins quatre lettres, le préfixe suivant sera l'un des quatre mots : $abaa$, $abab$, $baba$, $babb$, qui tous possèdent un facteur carré.

\Q
Un facteur propre de $s=s_0...s_{n-1}$ est de la forme $s_is_{i+1}...s_{j-1}$ avec $0\leq i < j \leq n$. On les examine tous à la recherche d'un facteur carré.
\medskip

On commence par définir une fonction testant si un facteur donné est un carré :

\lstinputlisting[linerange={1-11}]{\SourceFile}

Puis on teste tous les facteurs possibles :

\lstinputlisting[linerange={13-24}]{\SourceFile}

On peut majorer le nombre d'itérations par :
\[
\sum_{i=0}^{n-1}\sum_{j=i+1}^n\frac{j-i}{2}=\frac{1}{2}\sum_{i=0}^{n-i}j=\frac{1}{4}\sum_{i=0}^{n-1}(n-i)(n-i+1)=\frac{1}{4}\sum_{i=1}^ni(i+1)=\frac{n(n+1)(n+2)}{12}
\]

On peut réduire la complexité en remplaçant la comparaison effectuée dans \texttt{est\_carre} par une comparaison de deux nombres. Il s'agit de considérer les lettres comme des chiffres et le cardinal de l'alphabet comme la base.\\
Si on se limite à une longueur de répétition équivalente au plus grand entier machine, le nombre d'itérations devient de l'ordre de $n^2$ au lieu de $n^3$.

\Q
Il n'est pas utile de repartir du début. Chaque caractère d'indice $i$ engendre deux nouveaux caractères en fin de mot (indices $2i$ et $2i+1$).

\lstinputlisting[linerange={26-38}]{\SourceFile}

Montrons qu'il est impossible d'avoir une sous-suite de la forme $\omega\omega x$ où $\omega$ est un mot et $x$ la première lettre de $\omega$.
\smallskip

Nous pouvons rapidement vérifier que 111 et 000 n'ont pas d'antécédents.
\smallskip

Supposons que $\omega\omega x$ soit le plus petit mot possible de cette forme. Il suffit de rechercher les antécédents possibles et voir que ceux-ci sont de la même forme ou inexistants, ce qui contredit notre hypothèse.
\medskip

Nous pouvons remarquer que cette suite de $\{0,1\}$ est identique à celle obtenue en regardant la parité du nombre de 1 dans l'écriture binaire des entiers :
\smallskip

\begin{tabular}{|c|c|c|c|c|c|c|c|}
    \hline
    entier & 0 & 1 & 2 & 3 & 4 & 5 & 6\\
    \hline
    parité & 0 & 1 & 1 & 0 & 1 & 0 & 0\\
    \hline
\end{tabular}

\Q
L'alphabet $\Sigma=\{0,1,2\}$ est suffisant car on ne peut avoir plus de deux 1 consécutifs, on aurait sinon un facteur de la forme $\omega\omega x$.
\medskip

Supposons que la suite possède deux facteurs consécutifs égaux. Ces derniers ont pour origine un facteur de la forme $0\nu0\nu0$ où $\nu\in\Sigma^*$, autrement dit de la forme $\omega\omega x$.
\medskip

Nous considérons $\Sigma=\{0,1,2\}$ et le morphisme $\phi$ tel que :\\
$0 \rightarrow (010 \rightarrow 0101) \rightarrow 1...$\\
et\\
$1 \rightarrow (010 \rightarrow 011001) \rightarrow 21...$\\
et\\
$2 \rightarrow (0110 \rightarrow 01101001) \rightarrow 210...$
\bigskip

\Fin
