\renewcommand{\SourceFile}{1-parcours-de-tableaux/src/1-4.ml}

\vspace{16pt}
\section{Deuxième plus grand élément}

Soit $T$ un tableau unidimensionnel d'entiers de taille $n\geq 1$. On cherche à déterminer l'indice du deuxième plus grand élément de $T$ (celui qui viendrait en deuxième position si on rangeait les éléments de $T$ par ordre décroissant).

\Q
Écrire une fonction OCaml qui calcule l'indice du deuxième plus grand élément de $T$ en parcourant une fois le tableau $T$.\\
QUel nombre de comparaisons entre éléments du tableau effectue votre fonction ?

\Q
Pour imaginer une meilleure solution, penser à un tournoi de tennis. Le deuxième meilleur joueur n'est pas forcément le finaliste mais figure parmi les adversaires du gagnant. Pourquoi ?\\
Donner un algorithme qui utilise cette analogie, et déterminer le nombre de comparaisons entre élémentts du tableau qu'il effectue.\\
Écrire alors la fonction OCaml correspondante.

\Corrige

\Q
Première solution évidente :

\lstinputlisting[linerange={1-11}]{\SourceFile}

Les $n-2$ comparaisons \texttt{t.(i) > !max\_1} sont toujours effectuées. Les $n-2$ comparaisons \texttt{t.(i) > !max\_2} le sont aussi dans le pire cas d'un tableau décroissant. Avec les comparaisons implicites de \texttt{min} et \texttt{max} (une seule suffirait mais serait moins lisible), on arrive à $2n-3$, ce qui est intuitif : il faut $n-1$ comparaisons pour trouver le plus grand parmi $n$ (chacun sauf le plus grand doit avoir été comparé à plus grand que lui) et $n-2$ pour trouver le plus grand parmi les $n-1$ éléments restants (même argument).

\Q
C'est clair : tout autre joueur que le gagnant et ses adversaires malheureux est de classement $\geq 3$ puisqu'on connaît deux meilleurs joueurs que lui.
\medskip

Pour l'algorithme, on simule un tournoi de tennis. On transforme les éléments du tableau en feuilles puis on les fusionne en mettant l'indice du plus grand élément des deux à la racine. En itérant jusqu'à n'obtenir qu'un seul arbre, on crée effectivement un arbre de tournoi dont l'indice du vainqueur est la racine.
\medskip

On effectuera $n-1$ comparaisons pour trouver le gagnant (maximum) car chaque match élimine un joueur. On cherche alors le maximum parmi les $\lceil \log_2 n \rceil$ joueurs battus par le gagnant.
\medskip

On commence par définir un type d'arbre et une fonction utilitaire :

\lstinputlisting[linerange={13-16}]{\SourceFile}

\lstinputlisting[firstline=18]{\SourceFile}

\Fin
