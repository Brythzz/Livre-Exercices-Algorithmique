\renewcommand{\SourceFile}{1-parcours-de-tableaux/src/1-3.ml}

\vspace{16pt}
\section{Travail sur des tableaux}

On se donne un tableau $T$ d'entiers de taille $N$.

\Q
On se donne un entier $p<N$ et un entier $s$, et on recherche dans le tableau $T$ les indices $k$ tels que $\sum^{p+k}_{i=k}T[i]\geq s$. Écrire une fonction OCaml qui effectue cette recherche. Donner en fonction de $N$ et $p$ un ordre de grandeur du nombre de tests et d'opérations arithmétiques effectuées lors de l'exécution de votre fonction. Pouvez-vous l'améliorer ?

\Q
On suppose maintenant que $T[0] < T[1] < T[2] < ... < T[N-1]$. Pouvez-vous tenir compte de cette information pour obtenir une fonction plus rapide ?

\Q
Proposer une fonction qui affiche les triplets d'entiers $i,j,k$ avec $j<i<N$ tels que $i^2+j^2=k^2$. Pouvez-vous l'améliorer ?

\Corrige

\Q
Première idée : une fonction intuitive, mais quelque peu idiote

\lstinputlisting[linerange={1-10}]{\SourceFile}

Si on réalise que lorsque l'on passe de l'étape $k$ à l'étape $k+1$ dans l'algorithme précédent, deux termes seulement de la somme changent, on obtient la fonction suivante, plus rapide si $p$ est plus grand que 2.

\lstinputlisting[linerange={12-24}]{\SourceFile}

\Q
L'idée est bien sûr de procéder par dichotomie pour savoir à partir de quel rang on a $\sum^{p+k}_{i=k}T[i]\geq s$. On peut procéder comme suit :

\lstinputlisting[linerange={33-43}]{\SourceFile}

Plusieurs variantes permettant de faire moins d'additions en utilisant le \og truc \fg{} de la question 1 sont possibles.

\newpage

\Q
La première idée (idiote!) est de faire une boucle sur $i$, $j$ et $k$ en utilisant le fait que $i^2 < k^2=i^2+j^2 < 2\times i^2$ donc $i < k < \sqrt{2}\times i < 1,5\times i$ d'où $i+1 \leq k \leq \lfloor 1,5\times i \rfloor$ :

\lstinputlisting[linerange={45-55}]{\SourceFile}

on peut ensuite se dire que le seul $k$ qui a une chance de convenir est $\sqrt{i^2+j^2}$ (si c'est un entier !), ce qui donne :

\lstinputlisting[firstline=57]{\SourceFile}

\Fin
