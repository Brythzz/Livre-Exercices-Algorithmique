\renewcommand{\SourceFile}{1-parcours-de-tableaux/src/1-7.ml}

\section{Permutations}

On considère un ensemble $E=\{0,...,N-1\}$. On représente une permutation $p$ de $E$ par un tableau \texttt{p} de taille $N$ tel que l'image d'un élément $i$ de $E$ est \texttt{p.(i)}.

\Q
Écrire un algorithme qui, étant donné un tableau \texttt{p}, vérifie que \texttt{p} représente effectivement une permutation de $E$.

\Q
Écrire un algorithme qui décompose une permutation en cycles.

\Q
On ordonne les permutations par ordre lexicographique (l'ordre du dictionnaire). Par exemple, si $N=4$, $0123<0213<1230<2013$. Écrire un algorithme qui associe à chaque permutation \texttt{p} la permutation suivante dans l'ordre lexicographique (quand elle existe). On pourra utiliser le plus grand entier $i$ tel que $p(i)<p(i+1)$.

\Q
Écrire un algorithme qui énumère toutes les permutations de $E$.

\Corrige

\Q
Il serait particulièrement maladroit de tester successivement si tous les nombres de 0 à $N-1$ sont dans \texttt{p}. Une solution simple consiste à marquer les images rencontrées.

\lstinputlisting[linerange={1-6}]{\SourceFile}

\Q
Comme d'habitude, la bonne idée est de procéder comme on le ferait \og à la main \fg{}. On part de 0 et on regarde ses images successives par la permutation en marquant tous les entiers rencontrés. On s'arrête lorsque l'on retombre sur 0. Puis on recommence avec le premier entier non marqué.
\medskip

Exemple : Soit la permutation $\begin{pmatrix}
    0 & 1 & 2 & 3 & 4 & 5 \\
    4 & 1 & 0 & 5 & 2 & 3
\end{pmatrix}$.
On trouve un premier cycle (0~4~2). Le premier élément non marqué est 1, on trouve un second cycle (1). Le premier élément non marqué est 3, on trouve le troisième et dernier cycle (3~5).

\lstinputlisting[linerange={7-23}]{\SourceFile}

\Q
La difficulté est de comprendre comment on calcule le suivant d'une permutation dans l'orde lexicographique.
\medskip

\begin{tabular}{ l l }
    Exemple : & le suivant de 0 1 3 2 est 0 2 1 3 \\
     & le suivant de 0 2 1 3 est 0 2 3 1 \\
     & le suivant de 1 2 3 0 est 1 3 0 2
\end{tabular}
\medskip

Il faut donc chercher le plus grand entier $i$ tel que $p(i)$, ..., $p(N-1)$ contienne un élément plus grand que $p(i)$. Il est très facile de voir que cela revient à chercher, comme indiqué dans l'énoncé, le plus grand entier $i$ tel que $p(i)<p(i+1)$. Le suivant est alors obtenu en remplaçant $p(i)$ par le plus petit élément qui lui soit supérieur parmi $p.(i+1)$, ..., $p.(N-1)$. On complète la permutation en triant par ordre croissant les éléments restants.
\medskip

Exemple : Pour $p = 0~1~3~2$, le plus grand élément tel que $p(i)<p(i+1)$ est 1. On remplace 1 par le plus petit élément plus grand que 1 parmi 3 et 2, soit 2. On trie les nombres restants (2 et 3) par ordre croissant. On obtient ainsi 0 2 1 3.
\medskip

Pour éviter d'avoir une fonction de tri à écrire, on peut également marquer les éléments recontrés dans la recherche du plus grand i tel que $p(i)<p(i+1)$.
\medskip

Le programme de calcul de la permutation $q$ suivant dans l'ordre lexicographique une permutation $p$ fixée va suivre la méthode décrite ci-dessus. La fonction \texttt{max\_croit} renvoie le plus grand entier $i$ tel que $p(i)<p(i+1)$ si un tel entier existe, $-1$ sinon.

\lstinputlisting[linerange={25-31}]{\SourceFile}

La fonction \texttt{inf} renvoie le plus petit entier parmi \texttt{p.(k+1)}, ..., \texttt{p.(N-1)} qui est plus grand que \texttt{p.(k)}.

\lstinputlisting[linerange={33-40}]{\SourceFile}

On a finalement la fonction calculant la permutation suivant \texttt{p} :

\lstinputlisting[linerange={42-72}]{\SourceFile}

\Q
Il suffit d'appliquer l'algorithme de la question précédente de manière itérative à partir de la permutation identité.

\lstinputlisting[firstline=74]{\SourceFile}

\Fin
