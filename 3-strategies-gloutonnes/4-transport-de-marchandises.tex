\renewcommand{\SourceFile}{3-strategies-gloutonnes/src/3-4.ml}

\section{Transport de marchandises}

On dispose de divers modèles de camions (on considère tout d'abord que l'on dispose d'un nombre illimité de camions de chaque modèle !) d'une contenance de 1 tonne, 2 tonnes, 5 tonnes et 10 tonnes. On cherche de combien de façons on peut charger $n$ tonnes ($n > 1$) en utilisant ces camions. Par exemple, si $n=6$, il y a 5 façons possibles :
\begin{itemize}
    \item utiliser un camion de 5 tonnes et un camion de 1 tonne
    \item utiliser 3 camions de 2 tonnes
    \item utiliser 2 camions de 2 tonnes et 2 camions de 1 tonne
    \item utiliser 1 camion de 2 tonnes et 4 camions de 1 tonne
    \item utiliser 6 camions d'une tonne
\end{itemize}

\Q
Écrire une fonction OCaml qui calcule le nombre de possibilités.

\Q
Même chose en supposant qu'on ne dispose pas de plus de \texttt{max10} camions de 10 tonnes, \texttt{max5} camions de 5 tonnes, \texttt{max2} camions de 2 tonnes et \texttt{max1} camions d'une tonne.

\Q
Le stock de camions est à nouveau illimité dans chaque catégorie. Écrire une fonction OCaml qui donne la façon de charger $n$ tonnes qui utilise le moins de camions. Justifier votre réponse. Votre réponse marcherait-elle encore si la contenance des camions était différente ? Que dire si les contenances sont 1, $p$, $p^2$, ..., $p^k$, où $p$ est un entier supérieur ou égal à 2 ?

\Corrige

\Q
On donne la fonction suivante (où on compte implicitement le nombre de camions de 1 tonne) :

\lstinputlisting[linerange={1-12}]{\SourceFile}

\Q
Il suffit de changer les bornes des boucles \og for \fg{} et de vérifier qu'on ne dépasse pas le nombre de camions de 1 tonne.

\lstinputlisting[linerange={16-29}]{\SourceFile}

\Q
On peut bien sûr employer l'algorithme de la question 1 et mémoriser à chaque fois le nombre de camions mais cette approche est peu efficace car de complexité proportionnelle au nombre de possibilités.
\medskip

On utilise donc plutôt un algorithme glouton qui utilise le plus possible de camions de 10 tonnes (soit $\lfloor n/10 \rfloor$), puis le plus possible de camions de 5 tonnes (c'est à dire $\lfloor (n\ \textrm{mod}\,10)/5\rfloor$) et ainsi de suite.

\lstinputlisting[firstline=31]{\SourceFile}

Montrons que cet algorithme donne bien les valeurs recherchées. Pour ceci, montrons d'abord que le nombre de camions de 10 tonnes d'une solution optimale est nécessairement $\lfloor n/10 \rfloor$. Soit une solution optimale :
\begin{itemize}
    \item cette solution utilise au plus un camion de 5 tonnes (sinon on obtiendrait une meilleure solution en remplaçant 2 camions de 5 tonnes par un camion de 10 tonnes) ;
    \item cette solution utilise au plus 2 camions de 2 tonnes (sinon on pourrait remplacer 3 camions de 2 tonnes par un de 5 tonnes + un de 1 tonne) ;
    \item cette solution utilise au plus 1 camion de 1 tonne (sinon on pourrait remplacer 2 camions de 1 tonne par 1 camion de 2 tonnes).
\end{itemize}
Donc la charge totale constituée par les camions qui ne sont pas des camions de 10 tonnes est au plus de $1\times5 + 2\times2 + 1=10$ tonnes. Elle ne peut pas être exactement de 10 tonnes, sinon on pourrait remplacer le tout par un camion de 10 tonnes, elle est donc au plus de 9 tonnes. Donc le nombre de camions de 10 tonnes est bien de $\lfloor n/10 \rfloor$ et ce qui reste est bien $n$ mod 10. On poursuit le raisonnement sans difficulté avec les camions de 5 puis de 2 tonnes.
\medskip

Cet algorithme ne donne pas forcément la solution optimale avec d'autres modèles de camions : si on a des camions de 18, 7 et 1 tonnes, et si $n=21$ tonnes, alors l'algorithme glouton donne 1 camion de 18 tonnes et 3 camions de 1 tonne, alors que la meilleure solution est 3 camions de 7 tonnes.
\medskip

Si les valeurs des camions sont 1, $p$, $p^2$, ..., $p^k$, où $p$ est un entier supérieur ou égal à 2, l'algorithme glouton donne toujours une solution optimale. On montre comme précédemment que dans une solution optimale :
\begin{itemize}
    \item il y a au plus $(p-1)$ camions de $p^{k-1}$ tonnes (sinon on pourrait remplacer $p$ camions de $p^{k-1}$ tonnes par un camion de $p^k$ tonnes) ;
    \item il y a au plus $(p-1)$ camions de $p^{k-2}$ tonnes...
\end{itemize}
\newpage

Donc la charge totale des camions de charge strictement inférieure à $p^k$ est au plus $\sum_{i=0}^{k-1}(p-1)p^k=p^k-1$, donc le nombre de camions de charge $p^k$ doit être $\lfloor n/p^k \rfloor$, etc.
\bigskip

\Fin
