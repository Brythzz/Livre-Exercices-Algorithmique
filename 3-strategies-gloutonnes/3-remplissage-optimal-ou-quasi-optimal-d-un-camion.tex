\renewcommand{\SourceFile}{3-strategies-gloutonnes/src/3-3.ml}

\section{Remplissage optimal ou quasi-optimal d'un camion}

On dispose de $n$ marchandises, de poids respectifs $a_1$, $a_2$, ..., $a_n$, où les $a_i$ sont des entiers strictement positifs ordonnés dans le sens croissant ($a_1 \leq a_2 \leq ... \leq a_n$). On dispose d'un camion dont la charge maximale autorisée est $P$. On désire le charger avec certaines marchandises, de manière à obtenir le plus grand poids possible inférieur ou égal à $P$. On cherche donc la plus grande valeur inférieure ou égale à $P$ que l'on puisse obtenir en sommant les éléments d'un sous-ensemble de $\{a_1,a_2,...,a_n\}$.

\Q
Écrire une fonction OCaml résolvant le problème (on demande seulement le poids maximal et non l'ensemble correspondant). Quelle est la complexité de votre fonction dans le pire cas ?

\Q
Pour obtenir un résultat plus rapidement, on va simplifier le problème : on se fixe une valeur $\varepsilon>0$ donnée, et on cherche maintenant un sous-ensemble de $\{a_1,a_2,....a_n\}$ dont la somme $S$ des éléments est inférieure à $P$ et telle que si $S^*$ est la somme correspondant à la solution optimale, alors $S\geq S^*(1-\varepsilon)$. Proposer un programme résolvant ce problème. Pourquoi est-il plus efficace que le précédent ?

\Corrige

\Q
On donne la fonction récursive suivante :

\lstinputlisting[linerange={1-7}]{\SourceFile}

Dans le pire cas, la somme des $a_i$ est inférieure à $P$ et on construit un arbre de décision de taille $2^n$. On a donc une complexité exponentielle.
\smallskip

Il n'est pas difficile de modifier la fonction pour renvoyer l'ensemble d'objets correspondant au poids maximal.

\Q
Il suffit de modifier la fonction précédente de manière à ne considérer un élément que si l'élément précédemment ajouté est inférieur à cet élément multiplié par $(1-\varepsilon/n)$. On montre alors par récurrence sur $i$ que pour tout $\sigma$ appartenant à l'ensemble des sommes des parties de $a_1,...,a_i$, il existe un $\lambda$ appartenant aux objets déjà ajoutés tel que $(1-\varepsilon/n)^i\sigma \leq \lambda \leq \sigma$. On en déduit donc que si $S$ est le résultat donné par l'algorithme et $S^*$ l'optimum, que $(1-\varepsilon/n)^nS^* \leq S$, ce qui donne $(1-\varepsilon)S^* \leq S$. L'incidence sur la complexité provient du fait qu'à tout moment, le rapport entre deux éléments consécutifs considérés est supérieur ou égal à $1/(1-\varepsilon/n)$. Le nombre maximum d'éléments à ajouter est donc majoré par le plus grand $k$ tel que $1/(1-\varepsilon/n)^k \leq P$, ce qui donne $k \approx (n\log P)/\varepsilon$.
\bigskip

\Fin
