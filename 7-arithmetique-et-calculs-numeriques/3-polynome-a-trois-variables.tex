\renewcommand{\SourceFile}{7-arithmetique-et-calculs-numeriques/src/7-3.ml}

\section{Polynômes à trois variables}

On veut représenter des polynômes à coefficients entiers sur trois variables $X$, $Y$ et $Z$. Un monôme est un élément de la forme $aX^bY^cZ^d$ où $a$ est un entier et $b$, $c$, $d$ des entiers positifs ou nuls. L'entier $a$ est le coefficient du monôme et le triplet $(b,c,d)$ le degré du monôme. Un polynôme $P$ est un ensemble non vide de monômes. Dans un polynôme, on peut regrouper les monômes de même degré en faisant la somme de leurs coefficients (bien sûr, si cette somme est nulle, il est sans intérêt de faire apparaître le monôme correspondant dans l'écriture du polynôme). Un polynôme est dit réduit s'il contient au plus un monôme de degré fixé.

\Q
Modéliser la situation décrite et écrire un algorithme qui, étant donné un polynôme, calcule une représentation réduite de ce polynôme. Quel est le nombre de comparaisons entre monômes effectuées par votre algorithme ?

\Q
Donner une manière de ranger les polynômes dans la modélisation d'un polynôme afin qu'il existe un algorithme résolvant la question 1 en effectuant au plus $N$ comparaisons entre monômes, où $N$ est le nombre de monômes. Donner cet algorithme.

\Q
Un polynôme est dit symétrique si le monôme $aX^bY^cZ^d$ apparaît dans $P$ si et seulement si les monômes $aX^bY^dZ^c$, $aX^cY^bZ^d$, $aX^cY^dZ^b$, $aX^dY^bZ^c$ et $aX^dY^cZ^b$ apparaissent. Écrire un algorithme qui, étant donné un polynôme $P$, indique si ce polynôme est symétrique ou non.

\Q
Un monôme est dit unitaire si son coefficient est égal à 1. Donner un algorithme qui énumère tous les monômes unitaires de degré $(b,c,d)$ tels que $b+c+d=k$ fixé.

\Corrige

\Q
Une modélisation simple consiste à représenter un monôme par un 4-tuple d'entiers et un polynôme par un tableau de monômes (l'utilisation de types structurés permettrait une modélisation plus élégante mais ces types ne sont pas au programme). Nous sommes ainsi amenés à faire les déclarations suivantes :

\lstinputlisting[linerange={1-2}]{\SourceFile}

Le principe de l'algorithme consiste à prendre un monôme dans le polynôme et à chercher tous les monômes de même degré. Il ne faut pas oublier \og d'effacer \fg{} les monômes traités en mettant leurs coefficients à 0.

\lstinputlisting[linerange={4-26}]{\SourceFile}

Dans le pire cas, tous les monômes du polynôme sont de degrés distincts et le nombre de comparaisons entre monômes est $\frac{N(N-1)}{2}$.

\Q
Pour n'avoir qu'au plus $N$ comparaisons à effectuer, il suffit que les monômes de même degré soient rangés consécutivement dans le polynôme. C'est le cas par exemple dès qu'on définit un ordre sur le degré des monômes et que l'on range les monômes selon cet ordre. Sous cette hypothèse, l'algorithme devient alors :

\lstinputlisting[linerange={29-59}]{\SourceFile}

\Q
On va supposer que le polynôme considéré est réduit. On va prendre successivement les monômes et aller vérifier que tous les monômes obtenus par permutation du degré sont présents. On mettra à 0 les coefficients de tous les monômes traités.

\lstinputlisting[linerange={61-90}]{\SourceFile}

\Q
Cette question ne présente aucune difficulté particulière. Il suffit d'énumérer tous les monômes de coefficient 1 à l'aide de deux boucles \og for \fg{} convenablement imbriquées.

\lstinputlisting[firstline=92]{\SourceFile}

\Fin
