\renewcommand{\SourceFile}{7-arithmetique-et-calculs-numeriques/src/7-2.ml}

\section{Conversion d'écriture romaine-décimale}

On rappelle le système d'écriture des nombres utilisés par les romains. Il utilise les symboles I=1, V=5, X=10, L=50, C=100, D=500, M=1000. On rappelle quelques exemples dont le candidat pourra s'inspirer pour déduire les règles d'écriture des nombres :
\medskip

(I, II, III, IV, V, VI, VII, VIII, IX, X) = (1, 2, 3, 4, 5, 6, 7, 8, 9, 10).
\medskip

XXXIX = 39, XL = 40.
\medskip

(XLI, XLII, XLIII, XLIV, XLV, XLVI, XLVII, XLVIII, XLIX, L) = (41, 42, 43, 44, 45, 46, 47, 48 , 49, 50).
\medskip

En particulier, on ne peut jamais avoir quatre symboles identiques consécutifs. Un nombre $n$ est supposé codé dans un tableau d'entiers s'il est écrit dans le système décimal, le chiffre le plus significatif étant en tête du tableau, et dans un tableau de caractères s'il est écrit dans le système romain.

\Q
Quel est le plus grand nombre qui puisse être écrit ? Quel est le nombre dont l'écriture est la plus longue ?

\Q
Donner un algorithme qui prend en entrée un nombre écrit dans le système romain et donne en sortie son écriture dans le système décimal usuel. Écrire la fonction correspondante.

\Q
Donner un algorithme de conversion de décimal en romain. Écrire la fonction correspondante.

\Q
Proposer un algorithme d'addition de deux nombres dans le système romain.

\Q
Que pensez-vous du nombre moyen de caractères nécessaires pour écrire un nombre en romain ?

\Corrige

\Q
Le plus grand nombre ne peut avoir plus de trois M pour les milliers. Les centaines, dizaines et unités peuvent toutes valoir 9. La réponse est donc 3999, c'est-à-dire MMMCMXXCIX.
\medskip

Le nombre le plus long doit aussi avoir trois M correspondant au chiffre des milliers, mais ensuite chaque chiffre 8 correspond à quatre caractères, le maximum possible. La réponse est donc 3888, c'est-à-dire MMMDCCCLXXXVIII qui comporte 15 caractères.

\Q
Notons que chaque symbole correspond à une quantité précise qui doit être soit ajoutée, soit retranchée du total, suivant le symbole qui suit. On peut par exemple commencer par convertir chaque symbole en la quantité correspondante, puis calculer l'entier représenté en faisant une suite d'additions et de soustractions, puis convertir cet entier en tableau de chiffres.

\lstinputlisting[linerange={1-34}]{\SourceFile}

Notons que la conversion d'un entier en tableau de chiffres est un exercice standard qui doit être maîtrisé par les candidats. Une autre solution aurait consisté à lire le tableau de caractères romains, en séparant les parties utilisées pour le chiffre des milliers, celui des centaines, celui des dizaines et celui des unités. Les milliers sont faciles à convertir (il suffit de compter le nombre de M). Pour les autres chiffres, une même fonction peut être appliquée pour convertir les centaines, les dizaines ou les unités, en remplaçant M, D et C respectivement par C, L et X pour les dizaines et par X, V et I pour les unités. On obtiendrait ainsi une fonction de traduction directe sans passer par l'évaluation du nombre.

\Q
L'algorithme naturel consiste à lire chaque décimale et à écrire la suite de caractères correspondante. On peut appeler une fonction auxiliaire \texttt{chiffre} qui fera le même travail pour les centaines, les dizaines et les unités, en remplaçant les symboles adéquatement. Soit $k$ le chiffre à convertir, et $j$ la première case libre du tableau.
\medskip

On suppose que le nombre ici que le nombre à coder est inférieur à 3999 et donc qu'on peut prendre $|\texttt{d}| = 4$.

\lstinputlisting[linerange={36-59}]{\SourceFile}

\Q
Il n'existe en fait pas de méthode plus facile que de d'abord faire la conversion en entier. On va donc simplement utiliser les fonctions précédentes. On commence par définir une fonction \texttt{tab\_to\_int} qui permet de calculer la valeur d'un entier écrit en décimal :

\lstinputlisting[linerange={61-67}]{\SourceFile}

Puis on obtient :

\lstinputlisting[firstline=69]{\SourceFile}

\Q
Calculons la longueur moyenne d'un nombre compris entre 0 et 3999 quand il est écrit en chiffres romains.
\medskip

Le chiffre des milliers prend 0, 1, 2 ou 3 caractères donc la longueur de sa traduction est : 0 pour les 1000 premiers nombres, 1 pour les 1000 suivants, 2 pour les 1000 suivants et 3 pour les 1000 suivants, soit en moyenne 2,5.
\medskip

Le chiffre des centaines a une longueur moyenne de $(0+1+2+3+2+1+2+3+4+2)/10$, soit 2 caractères.
\medskip

Le calcul est le même pour les dizaines et pour les unités. Un nombre s'écrit donc en moyenne avec $1,5+3\times2=7,5$ caractères. En revanche, un nombre écrit en notation décimale et compris entre 0 et 3999 utilise utilise 1 chiffre pour les 10 premiers nombres, 2 pour les 90 suivants, 3 pour les 900 suivants et 4 pour les 3000 suivants, soit en moyenne : $(10\times1+90\times2+900\times3+3000\times4)/4000=3,7225$.
\medskip

Notons qu'on aurait pu ajouter une question subsidiaire plus difficile : proposer un algorithme qui teste si un tableau de caractères est bien une représentation valide d'un nombre en écriture romaine.
\bigskip

\Fin
