\documentclass[10pt, french]{book} % removed hidelinks param

\usepackage[T1]{fontenc}
\usepackage[utf8]{inputenc}
\usepackage[french]{babel}
\usepackage{microtype}

\usepackage{tikz}
\usetikzlibrary{positioning}

\usepackage{amsmath, amsfonts}
\usepackage{lmodern}
\fontfamily{lmr}\selectfont

\usepackage{geometry}
\geometry{
  headsep=4mm,
  papersize={17.5cm,24cm},
  top=18mm,
  bottom=18mm,
  left=26mm,
  right=18mm
}

\setlength{\parindent}{0pt}

\usepackage{colortbl}


%%% Style d'en-tête %%%

\usepackage{fancyhdr}
\fancypagestyle{plain}{\fancyhf{}\renewcommand{\headrulewidth}{0pt}}

\pagestyle{fancy}
\fancyhf{}
\setlength{\headheight}{\baselineskip}
% \renewcommand{\headrulewidth}{.6cm}

\renewcommand{\sectionmark}[1]{\markright{#1}}
\renewcommand{\chaptermark}[1]{\markboth{\chaptername~\thechapter~:~#1}{}}

\fancyhead[OL]{\it\nouppercase\rightmark}
\fancyhead[OR]{\it\thepage}

\fancyhead[EL]{\it\thepage}
\fancyhead[ER]{\it\nouppercase\leftmark}

% Pas d'en-tête sur les pages avant chaque chapitre
\makeatletter
\def\cleardoublepage{\clearpage\if@twoside \ifodd\c@page\else
    \hbox{}
    \thispagestyle{plain}
    \newpage
    \if@twocolumn\hbox{}\newpage\fi\fi\fi}
\makeatother \clearpage{\pagestyle{plain}\cleardoublepage}


%%% Style des titres %%%

\usepackage{titlesec}

\newlength\chapnumb
\setlength\chapnumb{3cm}

% Chapitres
\titleformat{\chapter}{}{}{0pt} {
  \raggedleft
  \fontsize{3.8cm}{0pt}\selectfont\thechapter\\
  \vspace*{3cm}
  \fontsize{25pt}{0pt}\sl\bfseries\selectfont
}
\titlespacing*{\chapter}{0pt}{0pt}{1cm}

% Parties
\titleformat{\section}{\sl\LARGE\bfseries}{\thesection}{\fontdimen2\font}{}


%%% Style des listes %%%

\frenchsetup{StandardItemLabels=true}

\usepackage{enumitem}
\setlist[itemize]{left=0pt..8mm}


%%% Style des blocs de code %%%

\usepackage{listingsutf8}

\lstalias{text}{}
% \newcommand{\ColorLang}{caml}
\newcommand{\ColorLang}{text}

\lstset{
  inputencoding=utf8/latin1,
  basicstyle=\ttfamily,
  basewidth=0.5em,
  language=\ColorLang,
  % frame=single
}

\newcommand{\SourceFile}{}


%%% Définitions de macros %%%

\newcounter{question}[section] % Remis à 0 à chaque nouvelle section
\newcommand{\Q}{
  \stepcounter{question}
  \vspace{.4cm}
  {\sffamily Question \arabic{question}}
  \vspace{.4cm}\\
}

\newcommand{\Corrige}{
  \setcounter{question}{0}
  \vspace{1cm}
  \centerline{\sffamily \rule[0.6ex]{2cm}{.1mm} CORRIGÉ \rule[0.6ex]{2cm}{.1mm}}
  \vspace{-.4cm}
}

\newcommand{\Epigraph}[1]{
  \og{\it#1}\fg
}

\newcommand{\Fin}{
  \centerline{$\star\star\star\star\star\star\star\star$}
}


\begin{document}
    \raggedbottom
    % \frontmatter
    % \input{cover.tex}

    % \tableofcontents

    \mainmatter

    \chapter{Parcours de tableaux}
    \Epigraph{Ordre, Permutations, Jeux.}

    \section{Jeu d'échecs}

Sur un échiquier, on représentera chaque case par ses coordonnées $(i, j)$, la case en bas à gauche étant de coordonnées $(0, 0)$. Sur un tel échiquier, en un coup, un cavalier peut se déplacer de la case $(i, j)$ vers celles d'entre les 8 positions suivantes qui correspondent effectivement à une case de l'échiquier (abscisse et ordonnée comprises entre 0 et 7) : $(i-2, j+1)$, $(i-1, j+2)$, $(i+1, j+2)$, $(i+2, j+1)$, $(i+2, j -1)$, $(i+1, j-2)$, $(i-1, j-2)$ et $(i-2,j-1)$.

\Q
Écrire une fonction OCaml qui donne toutes les cases accessibles en $p$ coups au plus à partir d'une case $(i_0, j_0)$.

\Q
Écrire une fonction OCaml qui indique si toutes les cases sont accessibles à partir d'une case $(i_0, j_0)$ donnée, et si oui, quel est le plus petit nombre de coups permettant d'atteindre à partir de cette case n'importe quelle autre case de l'échiquier.

\Corrige

\Q
Choisissions déjà la structure de données. Définissions un type\\

\texttt{type case = array[1..2] of integer}

Lorem ipsum dolor sit amet consectetur adipiscing elit

\Fin
    \section{Médiane d'un tableau}

On dispose d'un tableau $A$ de $n$ entiers distincts.

\Q
Écrire une fonction OCaml \texttt{echange (A:int array) (i:int) (j:int) : unit} qui échange les éléments d'indices $i$ et $j$ du tableau $A$.

\Q
Soient $g$ et $d$ deux entiers, $1\leq g\leq d \leq n$. Posons $\alpha=A[g]$. On désire effectuer une permutation des éléments de $A$ d'indice compris entre $g$ et $d$, qui soit telle qu'après la permutation il existe un entier $pivot$, $(g\leq pivot \leq d)$ vérifiant :

\begin{itemize}
    \item $A[\textit{pivot}]=\alpha$,
    \item pour tout $i$ conmpris entre $g$ et \textit{pivot}, $A[i]\leq\alpha$,
    \item pour tout $i$ conmpris entre $\textit{pivot}+1$ et $d$, $A[i]>\alpha$,
    \item les éléments de $A$ d'indice strictement inférieur à $g$ ou strictement supérieur à $d$ restent inchangés.
\end{itemize}

Par exemple, si $n=6$, si les éléments de $A$ sont 5, 8, 7, 3, 9, 15, si $g=2$ et $d=5$, les éléments de $A$ après la permutation seront 5, 3, 7, 8, 9, 15 ou 5, 7, 3, 8, 15, 9, ou... (il n'y a pas unicité des permutations possibles), et \textit{pivot} sera égal à 4. Écrire une fonction OCaml qui effectue la permutation et donne la valeur de \textit{pivot} sans utiliser un autre tableau que $A$. Combien d'affectations (c'est à dire d'instructions \og \texttt{<-} \fg) et de tests nécessite-t-elle?

\Q
On appelle \textit{médiane} de $A$ un couple $(i,\alpha)$ tel que $1\leq i\leq n$, $A[i]=\alpha$, et $E(n/2)$ éléments de $A$ sont inférieurs strictement à $\alpha$ ($E(x)$ est la partie entière de $x$). Proposer une fonction de calcul de la médiane de $A$ utilisant la fonction de la question 2.

\Corrige
\Q
La question 1 est élémentaire.

\Q
Pour la fonction donnant la permutation (c'est une fonction dite de \textit{partition}), on maintient deux indices $i$ et $j$ tels qu'à tout moment les éléments d'indice compris entre $g$ et $i$ sont tous inférieurs ou égaux à $\alpha$, tandis que les éléments d'indice compris entre $j$ et $d$ sont tous supérieurs ou égaux à $\alpha$.
    \renewcommand{\SourceFile}{1-parcours-de-tableaux/src/1-3.ml}

\vspace{16pt}
\section{Travail sur des tableaux}

On se donne un tableau $T$ d'entiers de taille $N$.

\Q
On se donne un entier $p<N$ et un entier $s$, et on recherche dans le tableau $T$ les indices $k$ tels que $\sum^{p+k}_{i=k}T[i]\geq s$. Écrire une fonction OCaml qui effectue cette recherche. Donner en fonction de $N$ et $p$ un ordre de grandeur du nombre de tests et d'opérations arithmétiques effectuées lors de l'exécution de votre fonction. Pouvez-vous l'améliorer ?

\Q
On suppose maintenant que $T[0] < T[1] < T[2] < ... < T[N-1]$. Pouvez-vous tenir compte de cette information pour obtenir une fonction plus rapide ?

\Q
Proposer une fonction qui affiche les triplets d'entiers $i,j,k$ avec $j<i<N$ tels que $i^2+j^2=k^2$. Pouvez-vous l'améliorer ?

\Corrige

\Q
Première idée : une fonction intuitive, mais quelque peu idiote

\lstinputlisting[linerange={1-10}]{\SourceFile}

Si on réalise que lorsque l'on passe de l'étape $k$ à l'étape $k+1$ dans l'algorithme précédent, deux termes seulement de la somme changent, on obtient la fonction suivante, plus rapide si $p$ est plus grand que 2.

\lstinputlisting[linerange={12-24}]{\SourceFile}

\Q
L'idée est bien sûr de procéder par dichotomie pour savoir à partir de quel rang on a $\sum^{p+k}_{i=k}T[i]\geq s$. On peut procéder comme suit :

\lstinputlisting[linerange={33-43}]{\SourceFile}

Plusieurs variantes permettant de faire moins d'additions en utilisant le \og truc \fg{} de la question 1 sont possibles.

\newpage

\Q
La première idée (idiote!) est de faire une boucle sur $i$, $j$ et $k$ en utilisant le fait que $i^2 < k^2=i^2+j^2 < 2\times i^2$ donc $i < k < \sqrt{2}\times i < 1,5\times i$ d'où $i+1 \leq k \leq \lfloor 1,5\times i \rfloor$ :

\lstinputlisting[linerange={45-55}]{\SourceFile}

on peut ensuite se dire que le seul $k$ qui a une chance de convenir est $\sqrt{i^2+j^2}$ (si c'est un entier !), ce qui donne :

\lstinputlisting[firstline=57]{\SourceFile}

\Fin

    \renewcommand{\SourceFile}{1-parcours-de-tableaux/src/1-4.ml}

\vspace{16pt}
\section{Deuxième plus grand élément}

Soit $T$ un tableau unidimensionnel d'entiers de taille $n\geq 1$. On cherche à déterminer l'indice du deuxième plus grand élément de $T$ (celui qui viendrait en deuxième position si on rangeait les éléments de $T$ par ordre décroissant).

\Q
Écrire une fonction OCaml qui calcule l'indice du deuxième plus grand élément de $T$ en parcourant une fois le tableau $T$.\\
QUel nombre de comparaisons entre éléments du tableau effectue votre fonction ?

\Q
Pour imaginer une meilleure solution, penser à un tournoi de tennis. Le deuxième meilleur joueur n'est pas forcément le finaliste mais figure parmi les adversaires du gagnant. Pourquoi ?\\
Donner un algorithme qui utilise cette analogie, et déterminer le nombre de comparaisons entre élémentts du tableau qu'il effectue.\\
Écrire alors la fonction OCaml correspondante.

\Corrige

\Q
Première solution évidente :

\lstinputlisting[linerange={1-11}]{\SourceFile}

Les $n-2$ comparaisons \texttt{t.(i) > !max\_1} sont toujours effectuées. Les $n-2$ comparaisons \texttt{t.(i) > !max\_2} le sont aussi dans le pire cas d'un tableau décroissant. Avec les comparaisons implicites de \texttt{min} et \texttt{max} (une seule suffirait mais serait moins lisible), on arrive à $2n-3$, ce qui est intuitif : il faut $n-1$ comparaisons pour trouver le plus grand parmi $n$ (chacun sauf le plus grand doit avoir été comparé à plus grand que lui) et $n-2$ pour trouver le plus grand parmi les $n-1$ éléments restants (même argument).

\Q
C'est clair : tout autre joueur que le gagnant et ses adversaires malheureux est de classement $\geq 3$ puisqu'on connaît deux meilleurs joueurs que lui.
\medskip

Pour l'algorithme, on simule un tournoi de tennis. On transforme les éléments du tableau en feuilles puis on les fusionne en mettant l'indice du plus grand élément des deux à la racine. En itérant jusqu'à n'obtenir qu'un seul arbre, on crée effectivement un arbre de tournoi dont l'indice du vainqueur est la racine.
\medskip

On effectuera $n-1$ comparaisons pour trouver le gagnant (maximum) car chaque match élimine un joueur. On cherche alors le maximum parmi les $\lceil \log_2 n \rceil$ joueurs battus par le gagnant.
\medskip

On commence par définir un type d'arbre et une fonction utilitaire :

\lstinputlisting[linerange={13-16}]{\SourceFile}

\lstinputlisting[firstline=18]{\SourceFile}

\Fin


    \chapter{Jouer avec les mots}
    \Epigraph{Reconnaissance, Cosntruction, Codage.}

    \chapter{Stratégies gloutonnes}
    \Epigraph{Le meilleur du moment, pour trouver le meilleur.}

    \chapter{Arborescences}
    \Epigraph{Un père, des fils. Une racine, des noeuds, des feuilles.}

    \chapter{Graphes}
    \Epigraph{Des arêtes, des sommets, des poids.}

    \chapter{Géométrie et Images}
    \Epigraph{Des représentations. Des figures. Des intersections. Des pavages.}

    \chapter[Arithmétique et calculs numériques]{Arithmétique et\linebreak calculs numériques}
    \Epigraph{Écrire. Calculer. Résoudre.}

    \chapter{Vers la récursivité}
    \Epigraph{Diviser pour régner. Faire moins pour faire plus.}

    \appendix

    \chapter{Un peu d'Algèbre}

    \section{Anneaux Booléens}

$F_2$ désignera le corps à deux éléments $\mathbb{Z}/2\mathbb{Z}$. Un \textit{anneau Booléen} est un anneau commutatif unitaire $(A, +, *)$ dans lequel, pour tout $x\in A, x*x=x$ ($F_2$ est un exemple d'anneau booléen).
\medskip

Si $E$ est un ensemble, on note $\mathcal{P}(E)$ l'ensemble des sous-ensembles finis de $E$ et $\mathcal{AB}[E]$ l'ensemble des fonctions de $\mathcal{P}(E)$ dans $F_2$ qui sont nulles sauf pour un nombre fini d'éléments de $\mathcal{P}(E)$. $\mathcal{AB}[E]$ est muni de l'addition et de la multiplication : $(f+g)(x)=f(x)+g(x)$ et $(fg)(x)=f(x)g(x)$. $\mathcal{AB}[E]$ muni de ces opérations est aussi un anneau booléen : c'est l'anneau booléen \textit{engendré} par $E$.
\medskip

Dans tout le problème, $E$ sera supposé fini et de cardinal $n$.

\Partie{Bases du calcul dans $\mathcal{AB}[E]$}

\begin{enumerate}
    \item Montrer que, dans tout anneau Booléen, pour tout $x$, $x+x=0$. En déduire que tout anneau booléen est naturellement muni d'une structure d'espace vectoriel sur $F_2$.
    \item À chaque $e \in E$ on associe la fonction $\chi_e \in \mathcal{AB}[E]$ telle que $\chi_e(S)=1$ si et seulement si $e \in S$. On appelle \textit{monôme} tout produit (fini) $\chi_{e_1}\dots\chi_{e_k}$ où $e_1$, ..., $e_k \in E$. Par convention, si $k=0$, on obtient le monôme 1. Montrer que l'ensemble des monômes est une base de $\mathcal{AB}[E]$. Préciser le nombre d'éléments de $\mathcal{AB}[E]$.
    \smallskip

    On appellera dans la suite \textit{b-polynôme} tout élément de $\mathcal{AB}[E]$ représenté comme somme de monômes distincts.
    \item On note $N=2^n$ et $E=\{1, ..., n\}$.\\
    Soit $C_1$ l'application qui à tout monôme $\chi_{i_1}\dots\chi_{i_k}$ ($i_1$, ..., $i_k$ étant distincts et $k \geq 1$) associe l'entier $2^{i_1-1} + \dots + 2^{i_k-1}$ et telle que $C_1(1)=0$. Montrer que $C_1$ est une bijection de l'ensemble des monômes dans $[0,2^n-1]$ et en déduire une représentation $C_1$ des éléments de $\mathcal{AB}[E]$ sous la forme d'un tableau de booléens de taille $N+1$. Implémenter en OCaml la somme et le produit de deux éléments de $\mathcal{AB}[E]$ ainsi représentés.
    \item À une partie $P$ non vide de $E$, on associe l'entier $C_2(P) = \sum_{p \in P}2^{p-1}$. Par convention, $C_2(\emptyset)=0$. Expliquer pourquoi $C_2$ est une bijection de $\mathcal{P}(E)$ dans $[0, 2^n-1]$. Montrer comment $C_2$ se \og prolonge \fg{} en une (autre) représentation $C_2$ de $\mathcal{AB}[E]$ sous la forme d'un tableau de booléens de taille $N+1$.\\
    Implémenter en OCaml la somme et le produit de deux éléments de $\mathcal{AB}[E]$ ainsi représentés. Comparer les deux représentation du point de vue du coût des opérations dans $\mathcal{AB}[E]$.
    \item Écrire une fonction OCaml qui permet de passer de $C_1(x)$ à $C_2(x)$ et qui implémente donc une fonction $\phi$ prenant en entrée un tableau de booléens de taille $N+1$ et renvoie un autre tableau de booléens de même taille. Implémenter aussi la réciproque de $\phi$.\\
    Application numérique : calculer $\phi(x)$ et $\phi^{-1}(x)$ pour les valeurs de $x$ suivantes (on suppose $n=3$) : $\chi_1$, $\chi_1\chi_2$, $\chi_1\chi_2+\chi_2+\chi_3$. Que remarquez-vous ?
\end{enumerate}

\Partie{Résolution d'équations dans $\mathcal{AB}[E \cup E']$. Méthode de Boole.}

Dans cette partie, pour améliorer la lisibilité, on notera $a=\chi_1$ et $b=\chi_2$.
\medskip

On considère ici une équation :
\begin{equation}\label{eq:1}
    f(x_1, ..., x_m) = 0
\end{equation}

\end{document}