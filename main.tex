\documentclass[10pt, french]{book} % removed hidelinks param

\usepackage[T1]{fontenc}
\usepackage[utf8]{inputenc}
\usepackage[french]{babel}
\usepackage{microtype}

\usepackage{tikz}
\usetikzlibrary{positioning}

\usepackage{amsmath, amsfonts}
\usepackage{lmodern}
\fontfamily{lmr}\selectfont

\usepackage{geometry}
\geometry{
  headsep=4mm,
  papersize={17.5cm,24cm},
  top=18mm,
  bottom=18mm,
  left=26mm,
  right=18mm
}

\setlength{\parindent}{0pt}

\usepackage{colortbl}


%%% Style d'en-tête %%%

\usepackage{fancyhdr}
\fancypagestyle{plain}{\fancyhf{}\renewcommand{\headrulewidth}{0pt}}

\pagestyle{fancy}
\fancyhf{}
\setlength{\headheight}{\baselineskip}
% \renewcommand{\headrulewidth}{.6cm}

\renewcommand{\sectionmark}[1]{\markright{#1}}
\renewcommand{\chaptermark}[1]{\markboth{\chaptername~\thechapter~:~#1}{}}

\fancyhead[OL]{\it\nouppercase\rightmark}
\fancyhead[OR]{\it\thepage}

\fancyhead[EL]{\it\thepage}
\fancyhead[ER]{\it\nouppercase\leftmark}


%%% Style des titres %%%

\usepackage{titlesec}

\newlength\chapnumb
\setlength\chapnumb{3cm}

% Chapitres
\titleformat{\chapter}{}{}{0pt} {
  \raggedleft
  \fontsize{3.8cm}{0pt}\selectfont\thechapter\\
  \vspace*{3cm}
  \fontsize{25pt}{0pt}\sl\bfseries\selectfont
}
\titlespacing*{\chapter}{0pt}{0pt}{1cm}

% Parties
\titleformat{\section}{\sl\LARGE\bfseries}{\thesection}{\fontdimen2\font}{}


%%% Style des listes %%%

\frenchsetup{StandardItemLabels=true}

\usepackage{enumitem}
\setlist[itemize]{left=0pt..8mm}


%%% Style des blocs de code %%%

\usepackage{listingsutf8}

\lstalias{text}{}
% \newcommand{\ColorLang}{caml}
\newcommand{\ColorLang}{text}

\lstset{
  inputencoding=utf8/latin1,
  basicstyle=\ttfamily,
  basewidth=0.5em,
  language=\ColorLang,
  % frame=single
}

\newcommand{\SourceFile}{}


%%% Définitions de macros %%%

\newcounter{question}[section] % Remis à 0 à chaque nouvelle section
\newcommand{\Q}{
  \stepcounter{question}
  \vspace{.4cm}
  {\sffamily Question \arabic{question}}
  \vspace{.4cm}\\
}

\newcommand{\Corrige}{
  \setcounter{question}{0}
  \vspace{1cm}
  \centerline{\sffamily \rule[0.6ex]{2cm}{.1mm} CORRIGÉ \rule[0.6ex]{2cm}{.1mm}}
  \vspace{-.4cm}
}

\newcommand{\Epigraph}[1]{
  \og{\it#1}\fg
}

\newcommand{\Fin}{
  \centerline{$\star\star\star\star\star\star\star\star$}
}


\begin{document}
    \raggedbottom
    % \frontmatter
    % \input{cover.tex}

    % \tableofcontents

    \mainmatter

    \chapter{Parcours de tableaux}
    \Epigraph{Ordre, Permutations, Jeux.}

    % \renewcommand{\SourceFile}{1-parcours-de-tableaux/src/1-1.ml}

\section{Jeu d'échecs}

Sur un échiquier, on représentera chaque case par ses coordonnées $(i, j)$, la case en bas à gauche étant de coordonnées $(0, 0)$. Sur un tel échiquier, en un coup, un cavalier peut se déplacer de la case $(i, j)$ vers celles d'entre les 8 positions suivantes qui correspondent effectivement à une case de l'échiquier (abscisse et ordonnée comprises entre 0 et 7) : $(i-2, j+1)$, $(i-1, j+2)$, $(i+1, j+2)$, $(i+2, j+1)$, $(i+2, j -1)$, $(i+1, j-2)$, $(i-1, j-2)$ et $(i-2,j-1)$.

\Q
Écrire une fonction OCaml qui donne toutes les cases accessibles en $p$ coups au plus à partir d'une case $(i_0, j_0)$.

\Q
Écrire une fonction OCaml qui indique si toutes les cases sont accessibles à partir d'une case $(i_0, j_0)$ donnée, et si oui, quel est le plus petit nombre de coups permettant d'atteindre à partir de cette case n'importe quelle autre case de l'échiquier.

\Corrige

\Q
Choisissons déjà la structure de données. Définissons un type

\lstinputlisting[linerange={1}]{\SourceFile}

et donnons-nous une fonction \texttt{est\_valide (c : case) : bool} qui renvoie \texttt{true} si $c$ est bien une case de l'échiquier (i.e $0\leq \texttt{fst c}$, $\texttt{snd c} \leq 7$), et \texttt{false} sinon. Réalisons tout de suite que quel que soit le nombre de coups, on ne pourra jamais atteindre plus de 64 cases et définissons des tableaux

\lstinputlisting[linerange={3-5}]{\SourceFile}

tel que \texttt{coups[i,j]} désigne la $j$ème des cases atteignables en exactement $i$ coups et \texttt{ncases[i]} le nombre de cases atteignables en exactement $i$ coups - et non atteignables en moins de coups. La seule difficulté de la fonction et de penser à \og faire le ménage \fg en vérifiant, à chaque fois qu'on ajoute une case, si elle n'a pas déjà été atteinte. Il est à noter qu'il suffit de vérifier ceci pour des valeurs de $i$ (numéros des coups) de même parité que le numéro courant : si on part d'une case blanche, en un nombre par de coups, on sera forcément sur une case blanche, et en un nombre impair sur une case noire.

\lstinputlisting[firstline=10]{\SourceFile}

\Q
Il n'y a pas grand chose à modifier : il suffit de remarquer qu'on a forcément atteint toutes les cases atteignables dès que l'ajout d'un nouveau coup n'apporte rien. Il suffit de modifier très légèrement la fonction précédente pour remplacer la boucle \og for \fg{} sur la variable \texttt{p\_courant} par une boucle \og while \fg{} où l'on s'arrête dès que \texttt{ncases[p\_courant]} vaut zéro. Il est alors facile de voir si toutes les cases sont atteintes (il faut simplement additionner les valeurs des \texttt{ncases[i]} pour voir si on trouve 64, surtout ne pas tester pour toute case si elle est dans le tableau \texttt{coups}, ceci serait trop long). Le lecteur pourra essayer d'évaluer le coût de ces fonctions. Sans réfléchir, on trouve quelque chose d'exponentiel en le nombre de coups, sauf si on se souvient que l'échiquier n'a que 64 cases !

\Fin

    % \section{Médiane d'un tableau}

On dispose d'un tableau $A$ de $n$ entiers distincts.

\Q
Écrire une fonction OCaml \texttt{echange (A:int array) (i:int) (j:int) : unit} qui échange les éléments d'indices $i$ et $j$ du tableau $A$.

\Q
Soient $g$ et $d$ deux entiers, $1\leq g\leq d \leq n$. Posons $\alpha=A[g]$. On désire effectuer une permutation des éléments de $A$ d'indice compris entre $g$ et $d$, qui soit telle qu'après la permutation il existe un entier $pivot$, $(g\leq pivot \leq d)$ vérifiant :

\begin{itemize}
    \item $A[\textit{pivot}]=\alpha$,
    \item pour tout $i$ conmpris entre $g$ et \textit{pivot}, $A[i]\leq\alpha$,
    \item pour tout $i$ conmpris entre $\textit{pivot}+1$ et $d$, $A[i]>\alpha$,
    \item les éléments de $A$ d'indice strictement inférieur à $g$ ou strictement supérieur à $d$ restent inchangés.
\end{itemize}

Par exemple, si $n=6$, si les éléments de $A$ sont 5, 8, 7, 3, 9, 15, si $g=2$ et $d=5$, les éléments de $A$ après la permutation seront 5, 3, 7, 8, 9, 15 ou 5, 7, 3, 8, 15, 9, ou... (il n'y a pas unicité des permutations possibles), et \textit{pivot} sera égal à 4. Écrire une fonction OCaml qui effectue la permutation et donne la valeur de \textit{pivot} sans utiliser un autre tableau que $A$. Combien d'affectations (c'est à dire d'instructions \og \texttt{<-} \fg) et de tests nécessite-t-elle?

\Q
On appelle \textit{médiane} de $A$ un couple $(i,\alpha)$ tel que $1\leq i\leq n$, $A[i]=\alpha$, et $E(n/2)$ éléments de $A$ sont inférieurs strictement à $\alpha$ ($E(x)$ est la partie entière de $x$). Proposer une fonction de calcul de la médiane de $A$ utilisant la fonction de la question 2.

\Corrige
\Q
La question 1 est élémentaire.

\Q
Pour la fonction donnant la permutation (c'est une fonction dite de \textit{partition}), on maintient deux indices $i$ et $j$ tels qu'à tout moment les éléments d'indice compris entre $g$ et $i$ sont tous inférieurs ou égaux à $\alpha$, tandis que les éléments d'indice compris entre $j$ et $d$ sont tous supérieurs ou égaux à $\alpha$.
    % \renewcommand{\SourceFile}{1-parcours-de-tableaux/src/1-3.ml}

\vspace{16pt}
\section{Travail sur des tableaux}

On se donne un tableau $T$ d'entiers de taille $N$.

\Q
On se donne un entier $p<N$ et un entier $s$, et on recherche dans le tableau $T$ les indices $k$ tels que $\sum^{p+k}_{i=k}T[i]\geq s$. Écrire une fonction OCaml qui effectue cette recherche. Donner en fonction de $N$ et $p$ un ordre de grandeur du nombre de tests et d'opérations arithmétiques effectuées lors de l'exécution de votre fonction. Pouvez-vous l'améliorer ?

\Q
On suppose maintenant que $T[0] < T[1] < T[2] < ... < T[N-1]$. Pouvez-vous tenir compte de cette information pour obtenir une fonction plus rapide ?

\Q
Proposer une fonction qui affiche les triplets d'entiers $i,j,k$ avec $j<i<N$ tels que $i^2+j^2=k^2$. Pouvez-vous l'améliorer ?

\Corrige

\Q
Première idée : une fonction intuitive, mais quelque peu idiote

\lstinputlisting[linerange={1-10}]{\SourceFile}

Si on réalise que lorsque l'on passe de l'étape $k$ à l'étape $k+1$ dans l'algorithme précédent, deux termes seulement de la somme changent, on obtient la fonction suivante, plus rapide si $p$ est plus grand que 2.

\lstinputlisting[linerange={12-24}]{\SourceFile}

\Q
L'idée est bien sûr de procéder par dichotomie pour savoir à partir de quel rang on a $\sum^{p+k}_{i=k}T[i]\geq s$. On peut procéder comme suit :

\lstinputlisting[linerange={33-43}]{\SourceFile}

Plusieurs variantes permettant de faire moins d'additions en utilisant le \og truc \fg{} de la question 1 sont possibles.

\newpage

\Q
La première idée (idiote!) est de faire une boucle sur $i$, $j$ et $k$ en utilisant le fait que $i^2 < k^2=i^2+j^2 < 2\times i^2$ donc $i < k < \sqrt{2}\times i < 1,5\times i$ d'où $i+1 \leq k \leq \lfloor 1,5\times i \rfloor$ :

\lstinputlisting[linerange={45-55}]{\SourceFile}

on peut ensuite se dire que le seul $k$ qui a une chance de convenir est $\sqrt{i^2+j^2}$ (si c'est un entier !), ce qui donne :

\lstinputlisting[firstline=57]{\SourceFile}

\Fin

    % \renewcommand{\SourceFile}{1-parcours-de-tableaux/src/1-4.ml}

\vspace{16pt}
\section{Deuxième plus grand élément}

Soit $T$ un tableau unidimensionnel d'entiers de taille $n\geq 1$. On cherche à déterminer l'indice du deuxième plus grand élément de $T$ (celui qui viendrait en deuxième position si on rangeait les éléments de $T$ par ordre décroissant).

\Q
Écrire une fonction OCaml qui calcule l'indice du deuxième plus grand élément de $T$ en parcourant une fois le tableau $T$.\\
QUel nombre de comparaisons entre éléments du tableau effectue votre fonction ?

\Q
Pour imaginer une meilleure solution, penser à un tournoi de tennis. Le deuxième meilleur joueur n'est pas forcément le finaliste mais figure parmi les adversaires du gagnant. Pourquoi ?\\
Donner un algorithme qui utilise cette analogie, et déterminer le nombre de comparaisons entre élémentts du tableau qu'il effectue.\\
Écrire alors la fonction OCaml correspondante.

\Corrige

\Q
Première solution évidente :

\lstinputlisting[linerange={1-11}]{\SourceFile}

Les $n-2$ comparaisons \texttt{t.(i) > !max\_1} sont toujours effectuées. Les $n-2$ comparaisons \texttt{t.(i) > !max\_2} le sont aussi dans le pire cas d'un tableau décroissant. Avec les comparaisons implicites de \texttt{min} et \texttt{max} (une seule suffirait mais serait moins lisible), on arrive à $2n-3$, ce qui est intuitif : il faut $n-1$ comparaisons pour trouver le plus grand parmi $n$ (chacun sauf le plus grand doit avoir été comparé à plus grand que lui) et $n-2$ pour trouver le plus grand parmi les $n-1$ éléments restants (même argument).

\Q
C'est clair : tout autre joueur que le gagnant et ses adversaires malheureux est de classement $\geq 3$ puisqu'on connaît deux meilleurs joueurs que lui.
\medskip

Pour l'algorithme, on simule un tournoi de tennis. On transforme les éléments du tableau en feuilles puis on les fusionne en mettant l'indice du plus grand élément des deux à la racine. En itérant jusqu'à n'obtenir qu'un seul arbre, on crée effectivement un arbre de tournoi dont l'indice du vainqueur est la racine.
\medskip

On effectuera $n-1$ comparaisons pour trouver le gagnant (maximum) car chaque match élimine un joueur. On cherche alors le maximum parmi les $\lceil \log_2 n \rceil$ joueurs battus par le gagnant.
\medskip

On commence par définir un type d'arbre et une fonction utilitaire :

\lstinputlisting[linerange={13-16}]{\SourceFile}

\lstinputlisting[firstline=18]{\SourceFile}

\Fin

    % \renewcommand{\SourceFile}{1-parcours-de-tableaux/src/1-5.ml}

\section{Tri d'un petit nombre d'éléments}

Soit $n$ un entier $\geq 2$. Le but de cet exercice est d'étudier les algorithmes de tri d'un tableau de n éléments entiers, pour $n=3,4,5$ et 10. On ne compte que les comparaisons entre éléments du tableau, et on note $Comp(n)$ leur nombre.

\Q
Donner un algorithme, et écrire la fonction OCaml correspondante, pour trier un tableau de taille $n=3$ avec $Comp(n)=3$.\\
Même question avec $n=4$ et $Comp(n)=5$.\\
Même question avec $n=5$ et $Comp(n)=7$.

\Q
On suppose que $n=10$. On demande de préciser le nombre de comparaisons pour chacun des deux algorithmes basés sur les principes suivants (mais on ne demande aucune fonction OCaml dans cette question) :
\medskip

Algorithme 1. Faire deux listes de 5 éléments, utiliser deux fois l'algorithme précédent pour $n=5$ et fusionner les deux listes triées obtenues.
\medskip

Algorithme 2. Faire 5 paires d'éléments et trier les 5 plus grands de chaque paire à l'aide de l'algorithme pour $n=5$. Puis insérer les éléments restants dans la liste formée.

\Corrige
\vspace{.6cm}

Commençons par définir une fonction qui range dans l'ordre deux éléments d'un tableau \texttt{t} en positions \texttt{i} et \texttt{j} :

\lstinputlisting[linerange={1-6}]{\SourceFile}

\Q
Très facile avec 3 éléments : on met le plus petit élément en position 1 en le comparant aux deux autres, puis on classe les deux éléments restants :

\lstinputlisting[linerange={8-11}]{\SourceFile}

Pour 4 éléments : on classe les paires en position (0, 1) et (2, 3), puis on met les deux plus petits en position 0 et le plus grand des deux plus grands en position 3. Enfin, reste à classer les éléments en positions 1 et 2 :

\lstinputlisting[linerange={13-18}]{\SourceFile}

Une autre solution est de trier les 3 premiers éléments (3 comparaisons) puis d'insérer le quatrième, ce qui coute 2 comparaisons supplémentaires en procédant par dichotomie, c'est-à-dire en commençant par l'élément du milieu.
\medskip

Avec 5 éléments : trier les 4 premiers éléments puis insérer le cinquième par dichotomie est ici trop coûteux : il faut 3 comparaisons pour insérer un élément dans une liste de taille 4. Cette solution nous coûterait $5+3=8$ comparaisons.
\medskip

Voici une solution en 7 comparaisons : on prend les 4 premiers éléments, on classe les paires (0, 1) et (2, 3) et on compare les deux plus grands éléments. La situation se résume à l'aide du diagramme (où $\texttt{a}\geq\texttt{b}\geq\texttt{c}$ et $\texttt{a}\geq\texttt{d}$) :

\begin{center}
    \begin{tikzpicture}[
        label distance=2pt,
        every node/.style={
            font=\ttfamily,
            draw,
            fill=black,
            circle,
            inner sep=0pt,
            minimum size=8pt}]
        \node[label=right:a] (A) at (1,2) {};
        \node[label=right:b] (B) at (2,1) {};
        \node[label=right:c] (C) at (2,0) {};
        \node[label=left:d] (D) at (1,1) {};
        \node[label=right:e] (E) at (4,1) {};

        \draw (A) -- (D);
        \draw (A) -- (B);
        \draw (B) -- (C);
    \end{tikzpicture}
\end{center}

On insère alors le cinquième élément $e$ dans la chaîne $a,b,c$ de longueur 3, ce qui coûte 2 comparaisons. Enfin, on insère le quatrième élément $d$ parmi les éléments inférieurs à $a$ dans l'ensemble $a,b,c,e$ ordonné, ce qui coûte deux comparaisons.
\medskip

On commence par définir une fonction utilitaire :

\lstinputlisting[firstline={20}]{\SourceFile}

La fonction est volumineuse mais ne pose pas de difficulté particulière.

\Q
Pour l'algorithme 1, le décompte est facile : on utilise deux fois la fonction \texttt{tri\_5}, ce qui conduit à 14 comparaisons. Reste à fusionner deux listes triées de 5 éléments. Or la fusion de deux listes triées de tailles respectives $m$ et $n$ requiert $m+n-1$ comparaisons, soit ici 9 comparaisons. On obtient un total de 23 comparaisons.
\medskip

Pour l'algorithme 2, après les 5 premières comparaisons et la fonction \texttt{tri\_5} appliquée aux 5 plus grands, on a la situation suivante :
\vspace{1.5cm}

\begin{center}
    \begin{tikzpicture}[
        label distance=2pt,
        every node/.style={
            font=\ttfamily,
            draw,
            fill=black,
            circle,
            inner sep=0pt,
            minimum size=8pt}]
        \node[label=right:a] (A) at (.75,4.75) {};
        \node[label=right:b] (B) at (.75,4) {};
        \node[label=right:c] (C) at (.75,3.25) {};
        \node[label=right:d] (D) at (.75,2.5) {};
        \node[label=right:e] (E) at (.75,1.75) {};

        \node[label=left:x] (X) at (0,4) {};
        \node[label=left:y] (Y) at (0,3.25) {};
        \node[label=left:z] (Z) at (0,2.5) {};
        \node[label=left:t] (T) at (0,1.75) {};
        \node[label=left:f] (F) at (0,1) {};


        \draw (A) -- (B);
        \draw (B) -- (C);
        \draw (C) -- (D);
        \draw (D) -- (E);
        \draw (A) -- (X);
        \draw (B) -- (Y);
        \draw (C) -- (Z);
        \draw (D) -- (T);
        \draw (E) -- (F);

    \end{tikzpicture}
\end{center}

\newpage

Reste à insérer les 4 éléments $x$, $y$, $z$ et $t$ dans la chaîne ordonnée de longueur 6 $f,e,d,c,b,a$. On commence par insérer $z$ dans $d,e,f$ (2 comparaisons) puis on insère $t$ dans la liste qui comprend $f$, $e$ et éventuellement $z$ (encore deux comparaisons). On a :
\bigskip

\begin{center}
    \begin{tikzpicture}[
        label distance=2pt,
        every node/.style={
            font=\ttfamily,
            draw,
            fill=black,
            circle,
            inner sep=0pt,
            minimum size=8pt}]
        \node[label=right:a] (A) at (.75,6.25) {};
        \node[label=right:b] (B) at (.75,5.5) {};
        \node[label=right:c] (C) at (.75,4.75) {};
        \node (S1) at (.75,4) {};
        \node (S2) at (.75,3.25) {};
        \node (S3) at (.75,2.5) {};
        \node (S4) at (.75,1.75) {};
        \node (S5) at (.75,1) {};

        \node[label=left:x] (X) at (0,5.5) {};
        \node[label=left:y] (Y) at (0,4.75) {};


        \draw (A) -- (B);
        \draw (B) -- (C);
        \draw (C) -- (S1);
        \draw (S1) -- (S2);
        \draw (S2) -- (S3);
        \draw (S3) -- (S4);
        \draw (S4) -- (S5);

        \draw (A) -- (X);
        \draw (B) -- (Y);

    \end{tikzpicture}
\end{center}
\bigskip

Insérer $x$ dans la chaîne de longueur 7 dont le plus grand élément est b requiert 3 comparaisons. Enfin, insérer $y$ dans une chaîne de longueur au plus 7 (longueur 6 si $x>b$, longueur 7 sinon) requiert également 3 comparaisons.
\medskip

Le nombre final de comparaisons est donc $5+7+2+2+3+3=22=Comp(10)$.
\medskip

Il se trouve que l'algorithme 2 est optimal. Bien sûr, c'est difficile à prouver ! D'une manière générale, la fonction $Comp\_min(n)$ qui donne le nombre minimal de comparaisons nécessaires pour trier $n$ éléments est très mal connue : on ne connait pas d'expression exacte, juste un équivalent asymptotique en $O(n\log_2n)$.
\bigskip

\Fin

    % \renewcommand{\SourceFile}{1-parcours-de-tableaux/src/1-6.ml}

\section{Représentation de permutations}

On dit qu'une permutation $p$ des entiers 0, 1, ..., $n$ est représentée sous forme normale si elle est stockée dans un tableau \texttt{p} tel que \texttt{p.(i)} contienne l'image de \texttt{i} par la permutation.

\Q
Écrire une fonction qui prend comme entrée une permutation $p$ sous forme normale et calcule le nombre de points fixes de $p$.

\Q
Écrire une fonction qui prend comme entrée une permutation $p$ sous forme normale et calcule le nombre de cycles de $p$.

\Q
On dit qu'une permutation est stockée sous forme de cycles si elle est stockée dans un tableau \texttt{c} construit comme suit : on stocke la permutation cycle par cycle, on regarde le plus petit élément de chaque cycle, on ordonne les cycles par ordre décroissant de leur plus petit élément. Par exemple, la permutation :

\begin{center}
    \begin{tabular}{ |c|c|c|c|c|c|c|c|c| }
        \hline
        i & 0 & 1 & 2 & 3 & 4 & 5 & 6 & 7 \\
        \hline
        \texttt{p.(i)} & 6 & 1 & 0 & 7 & 3 & 4 & 2 & 5 \\
        \hline
       \end{tabular}
\end{center}

a trois cycles, l'un réduit à 1, qui est un point fixe, l'autre de longueur 3, avec 0 qui a pour image 6 qui a pour image 2 qui a pour image 0, et le troisième de longueur 4, avec 4 qui a pour image 3 qui a pour image 7 qui a pour image 5 qui a pour image 4. Les plus petits éléments de ces cycles sont respectivement 1, 0 et 3. En écrivant d'abord le cycle qui contient 3, puis celui qui contient 1, puis celui qui contient 0, on obtient le tableau \texttt{c} :

\begin{center}
    \begin{tabular}{ |c|c|c|c|c|c|c|c|c| }
        \hline
        i & 0 & 1 & 2 & 3 & 4 & 5 & 6 & 7 \\
        \hline
        \texttt{c.(i)} & 3 & 7 & 5 & 4 & 1 & 0 & 6 & 2 \\
        \hline
       \end{tabular}
\end{center}

Écrire une fonction qui construit \texttt{c} à partir de \texttt{p}.

\Q
Montrer que l'application qui a un tableau \texttt{p} associe le tableau \texttt{c} est bijective. Proposer un algorithme qui prend comme entrée une permutation sous forme de cycles et donne en sortie la même permutation représentée sous forme normale. Écrire la fonction correspondante.

\Corrige

\Q
Fonction élémentaire.

\lstinputlisting[linerange={1-4}]{\SourceFile}

\Q
On peut par exemple parcourir les cycles un à un, en marquant les nombres comme \og vus \fg{} au fur et à mesure ; on repère qu'un cycle se termine lorsqu'on retombe sur le premier élément du cycle courant. Pour trouver le cycle suivant, on cherche simplement un nombre qui n'a pas encore été vu. La fonction requiert de manipuler deux boucles imbriquées, la boucle intérieure parcourant le cycle et la boucle extérieure allant de cycle en cycle.

\lstinputlisting[linerange={6-22}]{\SourceFile}

\Q
Un algorithme possible est de parcourir les cycles comme dans la question 2, en remplissant en même temps un tableau \og min \fg{} tel que \texttt{min[j]} contienne le plus petit élément du $j$-ième cycle. Puis on fait un deuxième parcours des cycles pour remplir le tableau de sortie : le premier cycle à parcourir sera celui tel que \texttt{min[j]} soit maximal, et ainsi de suite jusqu'à avoir écrit tous les cycles.

\lstinputlisting[linerange={24-64}]{\SourceFile}

\Q
La seule question pour savoir si la transformation de la question 3 est réversible est de décider comment séparer le tableau en cycles : ensuite, chaque cycle est écrit dans l'ordre ($i$, $p(i)$, $p(p(i))$, etc.), donc il est facile de reconstruire $p$. Pour faire la séparation des cycles, il suffit de remarquer que l'ordre dans lequel on a écrit les cycles, chaque cycle commençant par son plus petit élément et les cycles étant écrits dans l'ordre décroissant, assure que dans un tableau \texttt{c} donnant une permutation sous forme de cycles, $j=c(i)$ est le début d'un nouveau cycle si et seulement si $j$ est minimal parmi $c(0)$, $c(1)$, ..., $c(i)$. Donc l'application est bien inversible.
\medskip

La fonction de passage à une forme normale est en fait nettement plus simple que celle de la question 3. Il suffit de parcourir \texttt{c} en gardant en mémoire le minimum des valeurs vues jusqu'à présent, et en remarquant que $p(c(i))=c(i+1)$ sauf en bout de cycle.

\lstinputlisting[firstline=66]{\SourceFile}

\Fin

    % \renewcommand{\SourceFile}{1-parcours-de-tableaux/src/1-7.ml}

\section{Permutations}

On considère un ensemble $E=\{0,...,N-1\}$. On représente une permutation $p$ de $E$ par un tableau \texttt{p} de taille $N$ tel que l'image d'un élément $i$ de $E$ est \texttt{p.(i)}.

\Q
Écrire un algorithme qui, étant donné un tableau \texttt{p}, vérifie que \texttt{p} représente effectivement une permutation de $E$.

\Q
Écrire un algorithme qui décompose une permutation en cycles.

\Q
On ordonne les permutations par ordre lexicographique (l'ordre du dictionnaire). Par exemple, si $N=4$, $0123<0213<1230<2013$. Écrire un algorithme qui associe à chaque permutation \texttt{p} la permutation suivante dans l'ordre lexicographique (quand elle existe). On pourra utiliser le plus grand entier $i$ tel que $p(i)<p(i+1)$.

\Q
Écrire un algorithme qui énumère toutes les permutations de $E$.

\Corrige

\Q
Il serait particulièrement maladroit de tester successivement si tous les nombres de 0 à $N-1$ sont dans \texttt{p}. Une solution simple consiste à marquer les images rencontrées.

\lstinputlisting[linerange={1-6}]{\SourceFile}

\Q
Comme d'habitude, la bonne idée est de procéder comme on le ferait \og à la main \fg{}. On part de 0 et on regarde ses images successives par la permutation en marquant tous les entiers rencontrés. On s'arrête lorsque l'on retombe sur 0. Puis on recommence avec le premier entier non marqué.
\medskip

Exemple : Soit la permutation $\begin{pmatrix}
    0 & 1 & 2 & 3 & 4 & 5 \\
    4 & 1 & 0 & 5 & 2 & 3
\end{pmatrix}$.
On trouve un premier cycle (0~4~2). Le premier élément non marqué est 1, on trouve un second cycle (1). Le premier élément non marqué est 3, on trouve le troisième et dernier cycle (3~5).

\lstinputlisting[linerange={7-23}]{\SourceFile}

\Q
La difficulté est de comprendre comment on calcule le suivant d'une permutation dans l'ordre lexicographique.
\medskip

\begin{tabular}{ l l }
    Exemple : & le suivant de 0 1 3 2 est 0 2 1 3 \\
     & le suivant de 0 2 1 3 est 0 2 3 1 \\
     & le suivant de 1 2 3 0 est 1 3 0 2
\end{tabular}
\medskip

Il faut donc chercher le plus grand entier $i$ tel que $p(i)$, ..., $p(N-1)$ contienne un élément plus grand que $p(i)$. Il est très facile de voir que cela revient à chercher, comme indiqué dans l'énoncé, le plus grand entier $i$ tel que $p(i)<p(i+1)$. Le suivant est alors obtenu en remplaçant $p(i)$ par le plus petit élément qui lui soit supérieur parmi $p.(i+1)$, ..., $p.(N-1)$. On complète la permutation en triant par ordre croissant les éléments restants.
\medskip

Exemple : Pour $p = 0~1~3~2$, le plus grand élément tel que $p(i)<p(i+1)$ est 1. On remplace 1 par le plus petit élément plus grand que 1 parmi 3 et 2, soit 2. On trie les nombres restants (2 et 3) par ordre croissant. On obtient ainsi 0 2 1 3.
\medskip

Pour éviter d'avoir une fonction de tri à écrire, on peut également marquer les éléments rencontrés dans la recherche du plus grand i tel que $p(i)<p(i+1)$.
\medskip

Le programme de calcul de la permutation $q$ suivant dans l'ordre lexicographique une permutation $p$ fixée va suivre la méthode décrite ci-dessus. La fonction \texttt{max\_croit} renvoie le plus grand entier $i$ tel que $p(i)<p(i+1)$ si un tel entier existe, $-1$ sinon.

\lstinputlisting[linerange={25-31}]{\SourceFile}

La fonction \texttt{inf} renvoie le plus petit entier parmi \texttt{p.(k+1)}, ..., \texttt{p.(N-1)} qui est plus grand que \texttt{p.(k)}.

\lstinputlisting[linerange={33-40}]{\SourceFile}

On a finalement la fonction calculant la permutation suivant \texttt{p} :

\lstinputlisting[linerange={42-72}]{\SourceFile}

\Q
Il suffit d'appliquer l'algorithme de la question précédente de manière itérative à partir de la permutation identité.

\lstinputlisting[firstline=74]{\SourceFile}

\Fin


    \chapter{Jouer avec les mots}
    \Epigraph{Reconnaissance, Cosntruction, Codage.}

    \renewcommand{\SourceFile}{2-jouer-avec-les-mots/src/2-1.ml}

\section{Réécriture de mots}

On considère les mots écrits sur l'alphabet \{a, b, A, B\}, tel que $w=$abbaBabAA par exemple. Soit $\varepsilon$ le mot vide. On dit que deux mots $u$ et $v$ sont en relation si on peut réécrire des parties de $u$ de façon à obtenir $v$ après une suite de transformations effectuées en suivant les règles suivantes :

\begin{center}
    \begin{tabular}{ r c l }
        aA & $\rightarrow$ & $\varepsilon$ \\
        Aa & $\rightarrow$ & $\varepsilon$ \\
        $\varepsilon$ & $\rightarrow$ & aA \\
        $\varepsilon$ & $\rightarrow$ & Aa \\
        bB & $\rightarrow$ & $\varepsilon$ \\
        Bb & $\rightarrow$ & $\varepsilon$ \\
        $\varepsilon$ & $\rightarrow$ & bB \\
        $\varepsilon$ & $\rightarrow$ & Bb \\
        aab & $\rightarrow$ & baa \\
        baa & $\rightarrow$ & aab \\
        bba & $\rightarrow$ & abb \\
        abb & $\rightarrow$ & bba \\
    \end{tabular}
\end{center}

Par exemple :
\begin{center}
    \begin{tabular}{ r l }
         & aababaabAAABBB \\
        $\rightarrow$ & aababbaaAAABBBB \\
        $\rightarrow$ & aababbABBB \\
        $\rightarrow$ & aababbABBB \\
        $\rightarrow$ & aabbbaABBB \\
        $\rightarrow$ & aa \\
    \end{tabular}
\end{center}

\Q
Montrer que cette relation est une relation d'équivalence. Montrer que aa et AA commutent avec toutes les lettres.

\Q
Montrer que tout mot $w$ est équivalent à un mot $xyz$ (formé de trois mots $x$, $y$ et $z$ mis bout à bout), où : $x$ est de la forme aa...aa ou AA...AA et de longueur paire, $y$ est de la forme bb...bb ou BB...BB et de longueur paire, $z$ de la forme ababa...bab, ou ba...bab, ou ab...aba, ou ba...ba, c'est-à-dire alterne les lettres a et b, commençant par a ou b et finissant par a ou b. Un mot de la forme $xyz$ est dit canonique.

\Q
Proposer un codage des mots canoniques sous forme de triplets d'entiers.

\Q
Écrire une fonction \texttt{forme3 (w:string) : bool} qui prend en entrée un mot $w$ et teste si $w$ est un mot alternant les lettres a et b, c'est-à-dire de la forme de $z$.

\Q
Écrire une fonction \texttt{ajouter\_a (iu, ju, ku: int * int * int) : (int * int * int)} qui prend en entrée un mot canonique $u$, codé par un triplet \texttt{(iu, ju, ku)}, et donne en sortie le codage \texttt{(iv, jv, kv)} du mot $v=u$a.

\Q
Écrire une fonction \texttt{representant\_canonique (w:string) : (int * int * int)} \newline qui prend pour entrée un mot quelconque $w$ et donne en sortie un triplet \texttt{(i, j, k)} codant un mot canonique équivalent à $w$.

\Corrige

\Q
Réflexivité : il suffit de prendre une suite de transformations réduite à l'ensemble vide.\\
Symétrie : pour chaque règle, la transformation inverse est dans l'ensemble des règles donc toute suite de transformations peut être inversée.\\
Transitivité : évidente.\\
On a donc une relation d'équivalence.
\bigskip

Commutation de aa avec toutes les lettres : aa commute évidemment avec a.\\
aaA $\rightarrow$ a $\rightarrow$ Aaa, donc aa commute avec A.\\
aab $\rightarrow$ baa est une règle, donc aa commute avec b.\\
aaB $\rightarrow$ BbaaB $\rightarrow$ BaabB $\rightarrow$ Baa, donc aa commute avec B.
\bigskip

Commutation de AA avec toutes les lettres : AAa $\rightarrow$ A $\rightarrow$ aAA, donc AA commute avec a.\\
AA commute évidemment avec A.\\
AAb $\rightarrow$ AAbaaAA $\rightarrow$ AAaabAA $\rightarrow$ bAA, donc AA commute avec b.\\
AAB $\rightarrow$ AABaaAA $\rightarrow$ AAaaBAA $\rightarrow$ BAA, donc AA commute avec B.
\medskip

Remarquons que, de façon symétrique, bb et BB commutent également avec toutes les lettres.

\Q
Pour transformer $w$, on commence par bouger toutes les suites de deux lettres consécutives identiques et les mettre au début du mot, les aa et AA précédant les bb et BB, et par simplifier toutes les occurrences de aA, Aa, bB ou Bb. On se retrouve avec un mot commençant par des aa et AA, continuant avec des bb et BB, et se terminant avec un mot qui alterne des a ou A avec des b ou B. On remplace alors chaque A par AAa et chaque B par BBb, puis on ramène les AA et les BB plus au début : on se retrouve avec une suite de aa et de AA, suivie d'une suite de bb et de BB, suivie d'un mot de la forme de $z$. En simplifiant les aaAA, les AAaa, les bbBB et les BBbb, on obtient mot $xyz$ de la forme requise. Notons qu'il n'est pas demandé ici de montrer l'unicité de cette écriture.

\Q
On peut coder $x$ par un entier $i$, positif si x contient des aa et négatif s'il contient des AA, et de valeur absolue égale au nombre de lettres de $x$. De même, $y$ peut être codé par un entier $j$. On peut coder $z$ par un entier $k$, positif si $z$ commence par un a et négatif si $z$ commence par un b, et de valeur absolue égale au nombre de lettres de $z$.

\Q
Supposons la longueur de $w$ supérieure ou égale à 1. On regarde la première lettre de $w$, puis on teste si les suivantes alternes, en s'arrêtant dès qu'on trouve une erreur ou qu'on a parcouru tout le mot.

\lstinputlisting[linerange={1-11}]{\SourceFile}
\newpage

\Q
La fonction n'est pas difficicle mais demande un peu d'attention pour ne pas faire d'étourderie : il est particulièrement recommandé d'écrire d'abord l'algorithme en détail. Si $z$ est non vide et se termine par un b : \texttt{ku} strictement positif et pair ou strictement négatif et impair ; alors $z$ s'allonge d'une lettre : \texttt{ku} augmente de 1 dans le premier cas et diminue de 1 dans le deuxième cas. Si $z$ est non vide et se termine par un a : \texttt{ku} strictement positif et impair ou strictement négatif et pair ; alors $z$ raccourcit d'une lettre et $x$ change : \texttt{ku} diminue de 1 dans le premier cas et augmente de 1 dans le second cas, et \texttt{iu} augmente de 2 s'il était positif et diminue de 2 s'il était strictement négatif. Si $z$ est vide : $\texttt{ku}=0$ ; alors \texttt{ku} devient égal à 1. D'où la fonction suivante :

\lstinputlisting[linerange={13-23}]{\SourceFile}

\Q
On peut supposer qu'on dispose des fonctions \texttt{ajouter\_b}, \texttt{ajouter\_A} et \texttt{ajouter\_B} similaires à celle de la question 5. Il est alors facile d'ajouter des lettres une par une pour construire un mot canonique pour chaque préfixe de $w$ et finalement pour $w$.

\lstinputlisting[linerange={30-39}]{\SourceFile}
\medskip

La relation d'équivalence étudiée dans cet exercice correspond à un groupe donné par une permutation finie : les éléments sont les classes d'équivalence de la relation définie à la question 1, et on peut facilement vérifier que cette relation est compatible avec la concaténation. L'étude faite ici consiste en la construction d'une \og structure automatique \fg{} pour le groupe, permettant de calculer un représentant distingué de la classe du produit de deux éléments donnés. L'étude de groupes automatiques est un domaine de recherche récent et actif en mathématiques.
\bigskip

\Fin

    \renewcommand{\SourceFile}{2-jouer-avec-les-mots/src/2-2.ml}

\section{Pliage de papier}

On prend une feuille de papier et on la plie $n$ fois dans le sens vertical, en repliant à chaque fois la moitié droite sur la gauche. Les plis de la feuille, une fois redépliée, sont une suite de creux et de bosses. Voir figure :


\Q
Combien y a-t-il de plis à la n-ième étape ?

\Q
On représente chaque étape du pliage par un mot. Un creu est codé par un 0 et une bosse par un 1. Ainsi : $w_0=\varepsilon$ (mot vide)\\
$w_1=0$\\
$w_2=001$\\
$w_3=0010011$\\
\dots
\medskip

Montrer que $w_i$ est toujours un préfixe de $w_{i+1}$, c'est à dire que le début de $w_{i+1}$ coïncide avec $w_i$.

\Q
Donner un algorithme de construction de $w_n$ à partir de $w_{n-1}$. Écrire une fonction de calcul de $w_n$. On prendra comme entrée $n$ et on renverra une liste \texttt{w} d'entiers remplie adéquatement.

\Q
Écrire une fonction qui prend pour entrée un entier $n$ et renvoie une liste contenant la représentation binaire de $n$, poids fort en tête.

\Q
Les mots de la suite de pliage étant préfixes les uns des autres, on peut considérer le mot infini $w$ dont ils sont tous préfixes. Proposer un algorithme qui prend pour entrée la représentation binaire de $n$ et donne en sortie le $n$-ième bit de $w$.

\Corrige

\Q
Si on considère

    % \chapter{Stratégies gloutonnes}
    % \Epigraph{Le meilleur du moment, pour trouver le meilleur.}

    % \chapter{Arborescences}
    % \Epigraph{Un père, des fils. Une racine, des noeuds, des feuilles.}

    % \chapter{Graphes}
    % \Epigraph{Des arêtes, des sommets, des poids.}

    % \chapter{Géométrie et Images}
    % \Epigraph{Des représentations. Des figures. Des intersections. Des pavages.}

    % \chapter[Arithmétique et calculs numériques]{Arithmétique et\linebreak calculs numériques}
    % \Epigraph{Écrire. Calculer. Résoudre.}

    % \chapter{Vers la récursivité}
    % \Epigraph{Diviser pour régner. Faire moins pour faire plus.}

    % \appendix

    % \chapter{Un peu d'Algèbre}

    % \section{Anneaux Booléens}

$F_2$ désignera le corps à deux éléments $\mathbb{Z}/2\mathbb{Z}$. Un \textit{anneau Booléen} est un anneau commutatif unitaire $(A, +, *)$ dans lequel, pour tout $x\in A, x*x=x$ ($F_2$ est un exemple d'anneau booléen).
\medskip

Si $E$ est un ensemble, on note $\mathcal{P}(E)$ l'ensemble des sous-ensembles finis de $E$ et $\mathcal{AB}[E]$ l'ensemble des fonctions de $\mathcal{P}(E)$ dans $F_2$ qui sont nulles sauf pour un nombre fini d'éléments de $\mathcal{P}(E)$. $\mathcal{AB}[E]$ est muni de l'addition et de la multiplication : $(f+g)(x)=f(x)+g(x)$ et $(fg)(x)=f(x)g(x)$. $\mathcal{AB}[E]$ muni de ces opérations est aussi un anneau booléen : c'est l'anneau booléen \textit{engendré} par $E$.
\medskip

Dans tout le problème, $E$ sera supposé fini et de cardinal $n$.

\Partie{Bases du calcul dans $\mathcal{AB}[E]$}

\begin{enumerate}
    \item Montrer que, dans tout anneau Booléen, pour tout $x$, $x+x=0$. En déduire que tout anneau booléen est naturellement muni d'une structure d'espace vectoriel sur $F_2$.
    \item À chaque $e \in E$ on associe la fonction $\chi_e \in \mathcal{AB}[E]$ telle que $\chi_e(S)=1$ si et seulement si $e \in S$. On appelle \textit{monôme} tout produit (fini) $\chi_{e_1}\dots\chi_{e_k}$ où $e_1$, ..., $e_k \in E$. Par convention, si $k=0$, on obtient le monôme 1. Montrer que l'ensemble des monômes est une base de $\mathcal{AB}[E]$. Préciser le nombre d'éléments de $\mathcal{AB}[E]$.
    \smallskip

    On appellera dans la suite \textit{b-polynôme} tout élément de $\mathcal{AB}[E]$ représenté comme somme de monômes distincts.
    \item On note $N=2^n$ et $E=\{1, ..., n\}$.\\
    Soit $C_1$ l'application qui à tout monôme $\chi_{i_1}\dots\chi_{i_k}$ ($i_1$, ..., $i_k$ étant distincts et $k \geq 1$) associe l'entier $2^{i_1-1} + \dots + 2^{i_k-1}$ et telle que $C_1(1)=0$. Montrer que $C_1$ est une bijection de l'ensemble des monômes dans $[0,2^n-1]$ et en déduire une représentation $C_1$ des éléments de $\mathcal{AB}[E]$ sous la forme d'un tableau de booléens de taille $N+1$. Implémenter en OCaml la somme et le produit de deux éléments de $\mathcal{AB}[E]$ ainsi représentés.
    \item À une partie $P$ non vide de $E$, on associe l'entier $C_2(P) = \sum_{p \in P}2^{p-1}$. Par convention, $C_2(\emptyset)=0$. Expliquer pourquoi $C_2$ est une bijection de $\mathcal{P}(E)$ dans $[0, 2^n-1]$. Montrer comment $C_2$ se \og prolonge \fg{} en une (autre) représentation $C_2$ de $\mathcal{AB}[E]$ sous la forme d'un tableau de booléens de taille $N+1$.\\
    Implémenter en OCaml la somme et le produit de deux éléments de $\mathcal{AB}[E]$ ainsi représentés. Comparer les deux représentation du point de vue du coût des opérations dans $\mathcal{AB}[E]$.
    \item Écrire une fonction OCaml qui permet de passer de $C_1(x)$ à $C_2(x)$ et qui implémente donc une fonction $\phi$ prenant en entrée un tableau de booléens de taille $N+1$ et renvoie un autre tableau de booléens de même taille. Implémenter aussi la réciproque de $\phi$.\\
    Application numérique : calculer $\phi(x)$ et $\phi^{-1}(x)$ pour les valeurs de $x$ suivantes (on suppose $n=3$) : $\chi_1$, $\chi_1\chi_2$, $\chi_1\chi_2+\chi_2+\chi_3$. Que remarquez-vous ?
\end{enumerate}

\Partie{Résolution d'équations dans $\mathcal{AB}[E \cup E']$. Méthode de Boole.}

Dans cette partie, pour améliorer la lisibilité, on notera $a=\chi_1$ et $b=\chi_2$.
\medskip

On considère ici une équation :
\begin{equation}\label{eq:1}
    f(x_1, ..., x_m) = 0
\end{equation}

\end{document}