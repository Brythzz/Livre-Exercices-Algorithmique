\renewcommand{\SourceFile}{5-graphes/src/5-1.ml}

\section{Fête de Noël sans conflit}

On considère une grande famille de $n$ personnes avec beaucoup de gens qui ne s'entendent pas, représentée par une matrice $A=(a_{i,j})\in \mathscr{M}_n$ telle que $a_{i,j}=a_{j,i}=1$ si $i$ et $j$ ne peuvent pas se voir, 0 sinon. On a deux maisons de famille et on veut partager les $n$ personnes entre ces deux maisons pour les fêtes de Noël, de façon que deux personnes qui ne s'entendent pas soient toujours dans des maisons différentes.

\Q
Montrer par un exemple qu'il n'est pas toujours possible de répartir les membres de la famille entre les deux maisons pour éviter tout conflit.

\Q
On va mettre le résultat dans un tableau \texttt{maison} de taille $n$ tel que \texttt{maison.(i-1)=1} si la personne $i$ est dans la première maison et \texttt{maison.(i-1)=-1} si la personne $i$ est dans l'autre maison. Écrire une fonction \texttt{partage (a:int array array) : int array} qui teste s'il est possible de faire un partage sans conflit et propose un partage lorsque cela et possible.

\Q
Deux personnes $i$ et $j$ sont dites en relation d'influence s'il existe une suite $k_1$, ..., $k_l$ telle que $a_{i,k_1}=a_{k_1,k_2}=...=a_{k_{l-1},k_l}=a_{k_l,j}=1$. Montrer que cette relation est une relation d'équivalence. Soit $N$ le nombre de classes d'équivalence, si l'on pose par convention $a_{i,i}=1$ pour tout $i$. Montrer qu'un partage sans conflit est possible si et seulement si il n'existe pas de suite $i_1$, ..., $i_{2l+1}$ telle que $a_{i_1,i_2}=...=a_{i_{2l+1,i_1}}=1$. Montrer que le nombre de partages sans conflits est soit 0, soit $2N$.

\Q
On suppose maintenant qu'il y a trois maisons de famille. Proposer un algorithme qui fasse un partage sans conflits entre les trois maisons lorsque cela est possible. Qu'en pensez-vous ?

\Corrige

\Q
Il suffit d'une famille de trois personnes qui sont chacune en conflit avec les deux autres.
