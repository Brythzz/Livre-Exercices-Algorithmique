\usepackage[T1]{fontenc}
\usepackage[utf8]{inputenc}
\usepackage[french]{babel}
\usepackage{microtype}

\usepackage[hidelinks]{hyperref}

\usepackage{pdfpages}

\usepackage{tikz}
\usetikzlibrary{positioning, angles}

\usepackage{amsmath, amssymb, mathrsfs}
\usepackage{lmodern}
\fontfamily{lmr}\selectfont

\usepackage{geometry}
\geometry{
  headsep=4mm,
  papersize={17.5cm,24cm},
  top=18mm,
  bottom=18mm,
  left=26mm,
  right=18mm
}

\setlength{\parindent}{0pt}

\usepackage{colortbl}


%%% Style d'en-tête %%%

\usepackage{fancyhdr, ifthen, etoolbox}
\fancypagestyle{plain}{\fancyhf{}\renewcommand{\headrulewidth}{0pt}}

\pagestyle{fancy}
\fancyhf{}
\setlength{\headheight}{\baselineskip}

\renewcommand{\sectionmark}[1]{\markright{#1}}
\renewcommand{\chaptermark}[1]{\markboth{
  \ifstrequal{\thechapter}{0}
    {}
    {\chaptername~\thechapter~:~}#1}{}}

\fancyhead[OL]{\it\nouppercase\rightmark}
\fancyhead[OR]{\it\thepage}

\fancyhead[EL]{\it\thepage}
\fancyhead[ER]{\it\nouppercase\leftmark}

% Pas d'en-tête sur les pages avant chaque chapitre
\makeatletter
\def\cleardoublepage{\clearpage\if@twoside \ifodd\c@page\else
    \hbox{}
    \thispagestyle{plain}
    \newpage
    \if@twocolumn\hbox{}\newpage\fi\fi\fi}
\makeatother \clearpage{\pagestyle{plain}\cleardoublepage}


%%% Style des titres %%%

\usepackage{titlesec}

\newlength\chapnumb
\setlength\chapnumb{3cm}

% Ne pas afficher le numéro s'il vaut 0
\newcommand{\ChapterNumber}{
  \ifthenelse{\equal{\thechapter}{0}}
    {\vspace{2.4cm}}
    {\thechapter\\}
}

% Chapitres
\titleformat{\chapter}{}{}{0pt} {
  \raggedleft
  \fontsize{3.8cm}{0pt}\selectfont\ChapterNumber
  \vspace{3cm}
  \fontsize{25pt}{25pt}\sl\bfseries\selectfont
}
\titlespacing*{\chapter}{0pt}{0pt}{1cm}

% Parties
\titleformat{\section}{\sl\LARGE\bfseries}{\thesection}{\fontdimen2\font}{}


%%% Style de la table des matières %%%

\usepackage[titles]{tocloft}

% Format des titres de chapitres
\renewcommand{\cftchapfont}{\Large\itshape}
\renewcommand{\cftchappagefont}{\Large\itshape}
\renewcommand{\cftchapaftersnum}{\,:}
\setlength{\cftchapnumwidth}{\cftsecnumwidth}

\setlength{\cftbeforechapskip}{1.8em}
\renewcommand\cftchapafterpnum{\vskip4mm}

% Format des titres de sections
\setlength{\cftsecindent}{0em}
\setlength{\cftbeforesecskip}{0.5em}

%%% Style des listes %%%

\frenchsetup{StandardItemLabels=true}

\usepackage{enumitem}
\setlist{left=0pt..8mm}

%%% Style des blocs de code %%%

\usepackage{listingsutf8}

\lstalias{text}{}
% \newcommand{\ColorLang}{caml}
\newcommand{\ColorLang}{text}

\lstset{
  inputencoding=utf8/latin1,
  basicstyle=\ttfamily,
  basewidth=0.5em,
  language=\ColorLang,
  % frame=single
}

\newcommand{\SourceFile}{}


%%% Définitions de macros %%%

\newcounter{question}[section] % Remis à 0 à chaque nouvelle section
\newcommand{\Q}{
  \stepcounter{question}
  \vspace{.4cm}
  {\sffamily Question \arabic{question}}
  \vspace{.4cm}\\
}

\newcounter{partie}[section] % Remis à 0 à chaque nouvelle section
\newcommand{\Partie}[1]{
  \stepcounter{partie}
  \vspace{.4cm}
  {\sffamily \Roman{partie}. #1}
  \smallbreak
}

\newcommand{\Corrige}{
  \setcounter{question}{0}
  \vspace{1cm}
  \centerline{\sffamily \rule[0.6ex]{2cm}{.1mm} CORRIGÉ \rule[0.6ex]{2cm}{.1mm}}
  \vspace{-.4cm}
}

\newcommand{\Epigraph}[1]{
  \og{\it#1}\fg
}

\newcommand{\Fin}{
  \centerline{$\star\star\star\star\star\star\star\star$}
}
