\chapter{Préface}

Les exercices de cet ouvrage ont été proposés aux oraux des concours d'entrée des Écoles Normales Supérieures de Lyon et de la rue d'Ulm en 1994, 1995 et 1996. Beaucoup sont donnés dans leur forme originelle, d'autres ont été légèrement modifiés pour être un peu plus attractifs et accessibles.
\medskip

Ces exercices étaient généralement précédés de l'avertissement suivant :
\medskip

\textit{\og \ Le but de cet épreuve est de déterminer votre aptitude à\\
    - mettre en forme et analyser un problème\\
    - maîtriser les méthodes logiques propres à l'informatique\\
    - organiser et traiter des informations\\
    - rechercher, concevoir et mettre en forme un ou des algorithmes\\
    - construire méthodiquement un ou des programmes clairs\\
    - exposer de manière synthétique, claire et concise votre travail.}

\medskip
\textit{Le texte de l'épreuve est relativement succinct. Il vous est demandé, suivant votre convenance, de le compléter, pour décrire aussi précisément que possible, les limites d'utilisation de vos algorithmes et programmes. \fg{}}
\medskip

Les exercices sont regroupés par thème. L'étudiant pourra ainsi découvrir les grandes classes d'algorithmes et l'enseignant pourra trouver facilement des exemples pour illustrer un cours ou des travaux dirigés.
\medskip

Mis à part ceux de la rue d'Ulm, les exercices sont proposés avec un corrigé qui ne se veut pas un modèle du genre, qui propose une solution à une question posée. Les algorithmes proposés dans les corrigés sont écrits en OCaml mais aucun exercice n'est dépendant de ce langage de programmation ; ils peuvent tous facilement être retranscrits dans un autre langage. De plus, les parties en OCaml ont juste la prétention d'offrir une mise en forme lisible et rigoureuse des algorithmes proposés. Les fonctions proposées ne sont pas toujours orthodoxes, en particulier pour éviter certaines lourdeurs pouvant rendre la lecture difficile.
\newpage

Cet ouvrage est destiné à un public que l'on souhaite le plus large possible. Tout étudiant de classe préparatoire ou d'université désireux de s'exercer à l'algorithmique devrait y trouver satisfaction.
\medskip

Les auteurs.
\bigskip

\Fin
