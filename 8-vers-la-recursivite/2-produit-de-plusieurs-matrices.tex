\renewcommand{\SourceFile}{8-vers-la-recursivite/src/8-2.ml}

\section{Produit de plusieurs matrices}

On cherche à effectuer un produit de matrices
\[
    M_1 \times M_2 \times ... \times M_n
\]
où la matrice $i$ comporte $r_{i-1}$ lignes et $r_i$ colonnes, en effectuant la moins possible de multiplications de réels (la multiplication de deux matrices se fera par la méthode usuelle).

\Q
Écrire une fonction qui multiplie une matrice $A$ de taille $p \times q$ par une matrice $B$ de taille $q \times r$. Combien de multiplications de réels nécessite-t-elle ?

\Q
On effectue le produit $M_1 \times M_2 \times M_3 \times M_4$, où $M_1$ est de taille $10 \times 20$, $M_2$ de taille $20 \times 50$, $M_3$ de taille $50 \times 1$ et $M_4$ de taille $1 \times 100$. Combien de multiplications de réels fera-t-on si on calcule le produit dans les deux ordres (indiqués par les parenthèses) suivants :
\[
\begin{split}
    M_1 \times (M_2 \times (M_3 \times M_4))\\[6pt]
    (M_1 \times (M_2 \times M_3)) \times M_4
\end{split}
\]

\Q
Écrire une fonction pour trouver le plus petit nombre de multiplications de réels nécessaires pour calculer $M_0 \times M_1 \times ... \times M_{n-1}$, où la matrice $i$ comporte $r_{i-1}$ lignes et $r_i$ colonnes. Aide : définir une valeur \texttt{m.(i).(k)} égale au plus petit nombre de multiplications pour calculer $M_i \times M_{i+1} \times ... \times M_k$.

\Corrige

\Q
On a classiquement :

\lstinputlisting[linerange={1-14}]{\SourceFile}

Cette fonction utilise $p \times q \times r$ multiplications.

\Q
Il faut $50\times1\times100 + 20\times50\times100 + 10\times20\times100 = 125\,000$ multiplications pour la première méthode et $20\times50\times1 + 10\times20\times1 + 10\times1\times100 = 2200$ pour la seconde méthode.

\Q
Il suffit de voir comment s'obtient \texttt{m.(i).(j)}. On trouve aisément :
\[
    \texttt{m.(i).(j)} = \begin{cases}
        \ 0 & \textrm{si } i=j \\
        \ \min_{i \leq k < j}\ \{\texttt{m.(i).(k)} + \texttt{m.(k+1).(j)} + r_{i-1}r_kr_j\} & \textrm{sinon}
      \end{cases}
\]
Il n'est pas nécessaire d'écrire une fonction récursive, on itère sur la longueur des chaînes de multiplication.

\lstinputlisting[firstline=16]{\SourceFile}

\Fin
