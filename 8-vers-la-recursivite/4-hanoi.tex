\renewcommand{\SourceFile}{8-vers-la-recursivite/src/8-4.ml}

\section{Hanoï}

Les tours de Hanoï est un exemple très connu pour sa programmation récursive. Nous nous intéressons ici à sa version itérative.
\medskip

Le jeu en lui même est très simple. Nous disposons de trois colonnes sur lesquelles sont empilés des disques de taille différentes.
\medskip

Les règles sont les suivantes : sur chaque colonne, un disque plus gros ne peut pas être posé sur un disque plus petit, on ne peut déplacer qu'un seul disque à la fois.
\medskip

Au départ, tous les disques sont empilés sur la première colonne. Le joueur doit en un minimum de coups déplacer tous les disques de la première colonne à la dernière colonne.

\Q
Donner les déplacements à faire avec trois disques. Puis avec quatre disques.

\Q
Montrer que le plus petit disque se déplace de la même façon un coup sur deux. Montrer que pour chaque autre déplacement, un seul choix est possible.

\Q
Numérotons les coups en partant de 1.
\medskip

Déterminer la relation qu'il y a entre le numéro du disque à déplacer et le numéro du coup.
\medskip

Pour $n$ disques, combien de déplacements effectuez-vous ? Est-ce optimal ?
\medskip

Écrire le programme affichant tous les coups effectués.

\Corrige

\Q
Nous donnons dans les tableaux suivants les positions de chaque disque au cours du temps. Les colonnes sont numérotées 0, 1 et 2. Les disques sont nommés du plus petit au plus grand $d_0$, $d_1$, ..., $d_n$. Tous les disques sont sur la colonne 0 au départ.
\medskip

Pour trois disques nous avons :
\medskip

\begin{tikzpicture}[scale=1.5]
    \draw[thick] (0,0) -- (8,0);

    \draw[thick] (2,0) -- (2,2);
    \draw[thick] (4,0) -- (4,2);
    \draw[thick] (6,0) -- (6,2);

    \draw[draw] (1.75,.4) rectangle ++(.5,.2);
    \draw[draw] (1.5,.2) rectangle ++(1,.2);
    \draw[draw] (1.25,0) rectangle ++(1.5,.2);
\end{tikzpicture}
\medskip

Comme nous devons déplacer le plus grand disque de la colonne 0 à la colonne 2, la configuration suivante est un passage obligé :
\medskip

\begin{tikzpicture}[scale=1.5]
    \draw[thick] (0,0) -- (8,0);

    \draw[thick] (2,0) -- (2,2);
    \draw[thick] (4,0) -- (4,2);
    \draw[thick] (6,0) -- (6,2);

    \draw[draw] (3.75,.2) rectangle ++(.5,.2);
    \draw[draw] (3.5,0) rectangle ++(1,.2);
    \draw[draw] (1.25,0) rectangle ++(1.5,.2);
\end{tikzpicture}
\medskip

Pour obtenir cette configuration, le disque immédiatement inférieur au plus grand a dû être déplacé de la colonne 0 sur la colonne 1 d'où un autre passage obligé :
\medskip

\begin{tikzpicture}[scale=1.5]
    \draw[thick] (0,0) -- (8,0);

    \draw[thick] (2,0) -- (2,2);
    \draw[thick] (4,0) -- (4,2);
    \draw[thick] (6,0) -- (6,2);

    \draw[draw] (5.75,0) rectangle ++(.5,.2);
    \draw[draw] (1.5,.2) rectangle ++(1,.2);
    \draw[draw] (1.25,0) rectangle ++(1.5,.2);
\end{tikzpicture}
\medskip

Nous en déduisons le tableau suivant :
\medskip

\begin{tabular}{| c || c | c | c | c | c | c | c | c |}
    \hline
    coups & 0 & 1 & 2 & 3 & 4 & 5 & 6 & 7 \\
    \hline
    $d_0$ & 0 & 2 & 2 & 1 & 1 & 0 & 0 & 2 \\
    \hline
    $d_1$ & 0 & 0 & 1 & 1 & 1 & 1 & 2 & 2 \\
    \hline
    $d_2$ & 0 & 0 & 0 & 0 & 2 & 2 & 2 & 2 \\
    \hline
\end{tabular}
\medskip

Pour quatre disques, nous avons, en suivant le même raisonnement :
\medskip

\begin{tabular}{| c || c | c | c | c | c | c | c | c | c | c | c | c | c | c | c | c |}
    \hline
    coups & 0 & 1 & 2 & 3 & 4 & 5 & 6 & 7 & 8 & 9 & 10 & 11 & 12 & 13 & 14 & 15 \\
    \hline
    $d_0$ & 0 & 1 & 1 & 2 & 2 & 0 & 0 & 1 & 1 & 2 & 2 & 0 & 0 & 1 & 1 & 2 \\
    \hline
    $d_1$ & 0 & 0 & 2 & 2 & 2 & 2 & 1 & 1 & 1 & 1 & 0 & 0 & 0 & 0 & 2 & 2 \\
    \hline
    $d_2$ & 0 & 0 & 0 & 0 & 1 & 1 & 1 & 1 & 1 & 1 & 1 & 1 & 2 & 2 & 2 & 2 \\
    \hline
    $d_3$ & 0 & 0 & 0 & 0 & 0 & 0 & 0 & 0 & 2 & 2 & 2 & 2 & 2 & 2 & 2 & 2 \\
    \hline
\end{tabular}

\Q
Les deux exemples précédents mettent en évidence une dépendance entre la parité du nombre de disques et le sens de déplacement de ceux-ci.
\medskip

Tous les coups impairs, le disque $d_0$ se déplace de la colonne $x$ vers la colonne $x+(-1)^n$ mod 3, où $n$ représente le nombre de disques.
\medskip

Comme on ne peut rien mettre sur le disque $d_0$, le seul choix possible est de déplacer le plus petit des disques situés sur les deux autres colonnes. Maintenant, seul le disque $d_0$ est déplaçable sinon il y a retour à la situation précédente. Nous vérifions ainsi que $d_0$ est à déplacer un coup sur deux.

\Q
En fait, le disque $d_1$ se déplace comme s'il était le disque 0 des $n-1$ disques restants en ne prenant en compte que les coups pairs. Ce qui revient à un déplacement tous les coups pairs (multiples de deux) non multiples de quatre.
\medskip

Ainsi, le disque $d_i$ se déplace de $(-1)^{n-1}$ tous les coups multiples de $2^i$ non multiples de $2^{i+1}$.
\medskip

Ainsi, le disque $d_{n-1}$ se déplace au coup $2^{n-1}$ lorsque tous les autres disques sont sur la colonne 1. Le temps de les replacer sur la colonne 2 est le même que celui pour les placer sur la colonne 1, c'est-à-dire $2^{n-1}-1$.
\medskip

D'où le temps total pour $n$ disques : $2^n-1$. On peut maintenant le montrer par récurrence.
\medskip

Est-il optimal ? Oui, car pour déplacer le plus grand des disques, il faut avoir déplacé tous les autres sur l'autre colonne, ce qui donne la formule de récurrence suivante, où $T(n)$ représente le nombre de coups pour $n$ disques :
\[
    T(n) = 2\cdot T(n-1)+1
\]
Nous obtenons $T(n)=2^n-1$.
\medskip

Voici un exemple de programme indiquant les coups à jouer :

\lstinputlisting{\SourceFile}

La complexité en termes de déplacements est optimale. On peut calculer le nombre de comparaisons de la façon suivante (les seules comparaisons effectuées sont dans la condition de la boucle \texttt{while}) :
\begin{itemize}
    \item La moitié des nombres binaires se terminent par un 1. Dans ce cas, on effectue une comparaison.
    \item Parmi ceux se terminant par un 0, la moitié possède un 1 en deuxième position. Dans ce cas, on effectue 2 comparaisons.
    \item Parmi ceux se terminant par deux zéros, la moitié possède un 1 en troisième position. Dans ce cas, on effectue 3 comparaisons.
    \item De manière récurrente, on a donc une proportion $\frac{1}{2^n}$ des nombres qui nécessitent $n$ comparaisons.
\end{itemize}
\medskip

Donc en sommant tous les cas possibles, le nombre de comparaisons $N$ est donné par la formule :
\[
    N = 2^n\cdot\sum_{k=1}^nk\frac{1}{2^k}=2^{n-1}\cdot\sum_{k=1}^nk\left(\frac{1}{2}\right)^{k-1}
\]

En remarquant que :
\[
    \sum_{k=1}^nkx^{k-1} = \frac{\textrm{d}}{\textrm{d}x}\sum_{k=1}^nx^k = \frac{\textrm{d}}{\textrm{d}x}\frac{x-x^{n+1}}{1-x}
\]
on obtient $N = 2^{n+1}-(n+2)$.
\medskip

On peut si on le désire également donner la version récursive suivante où $A$, $B$ et $C$ représentent les trois colonnes :
\medskip

\texttt{Hanoï}($n$, $A$, $B$, $C$)\\
\hspace*{2em} si $n \geq 1$ faire\\
\hspace*{4em}   \texttt{Hanoï}($n-1$, $A$, $C$, $B$)\\
\hspace*{4em}   \textit{Déplacer le disque de $A$ vers $C$}\\
\hspace*{4em}   \texttt{Hanoï}($n-1$, $B$, $A$, $C$)
\bigskip

\Fin
